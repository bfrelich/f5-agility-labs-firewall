%% Generated by Sphinx.
\def\sphinxdocclass{report}
\documentclass[letterpaper,10pt,english]{sphinxmanual}
\ifdefined\pdfpxdimen
   \let\sphinxpxdimen\pdfpxdimen\else\newdimen\sphinxpxdimen
\fi \sphinxpxdimen=.75bp\relax

\usepackage[utf8]{inputenc}
\ifdefined\DeclareUnicodeCharacter
 \ifdefined\DeclareUnicodeCharacterAsOptional\else
  \DeclareUnicodeCharacter{00A0}{\nobreakspace}
\fi\fi
\usepackage{cmap}
\usepackage[T1]{fontenc}
\usepackage{amsmath,amssymb,amstext}
\usepackage{babel}
\usepackage{times}
\usepackage[Bjornstrup]{fncychap}
\usepackage[dontkeepoldnames]{sphinx}

\usepackage{geometry}

% Include hyperref last.
\usepackage{hyperref}
% Fix anchor placement for figures with captions.
\usepackage{hypcap}% it must be loaded after hyperref.
% Set up styles of URL: it should be placed after hyperref.
\urlstyle{same}
\addto\captionsenglish{\renewcommand{\contentsname}{Contents:}}

\addto\captionsenglish{\renewcommand{\figurename}{Fig.}}
\addto\captionsenglish{\renewcommand{\tablename}{Table}}
\addto\captionsenglish{\renewcommand{\literalblockname}{Listing}}

\addto\extrasenglish{\def\pageautorefname{page}}

\setcounter{tocdepth}{1}

%% LaTeX preamble.

\usepackage{type1cm}
\usepackage{helvet}
\usepackage{wallpaper}

% Bypass unicode character not supported errors
\usepackage[utf8]{inputenc}

\makeatletter
\def\UTFviii@defined#1{%
  \ifx#1\relax
      ?%
  \else\expandafter
    #1%
  \fi
}
\makeatother

\pagestyle{plain}
\pagenumbering{arabic}

\renewcommand{\familydefault}{\sfdefault}

\definecolor{f5red}{RGB}{235, 28, 35}

\def\frontcoverpage{
  \begin{titlepage}
  \ThisURCornerWallPaper{1.0}{front_cover}
  \vspace*{2.5cm}
  \hspace{4.5cm}
  {\color{f5red} \text{\Large Agility 2017 Hands-on Lab Guide}\par}
  \vspace{.5cm}
  \hspace{4.5cm}
  {\color{white} \text{\huge F5 Firewall Solutions}\par}
  \vspace{0.5cm}
  \hspace{4.5cm}
  {\color{white} \text{\large F5 Networks, Inc.}\par}
  \vfill
  \end{titlepage}
  \newpage
}

\def\backcoverpage{
  \newpage
  \thispagestyle{empty}
  \phantom{100}
  \ThisURCornerWallPaper{1.0}{back_cover}
}

\def\contentspage{
    \tableofcontents
}

%% Disable standard title (but keep PDF info).
\renewcommand{\maketitle}{
  \begingroup
  % These \defs are required to deal with multi-line authors; it
  % changes \\ to ', ' (comma-space), making it pass muster for
  % generating document info in the PDF file.
  \def\\{, }
  \def\and{and }
  \pdfinfo{
    /Title (F5 Firewall Solutions)
    /Author (F5 Networks, Inc.)
  }
  \endgroup
}


\title{F5 Firewall Solutions Documentation}
\date{Jul 04, 2017}
\release{}
\author{F5 Networks, Inc.}
\newcommand{\sphinxlogo}{\vbox{}}
\renewcommand{\releasename}{Release}
\makeindex

\begin{document}

\maketitle

\frontcoverpage
\contentspage

\phantomsection\label{\detokenize{index::doc}}



\chapter{Getting Started}
\label{\detokenize{intro:getting-started}}\label{\detokenize{intro::doc}}
Please follow the instructions provided by the instructor to start your
lab and access your jump host.

\begin{sphinxadmonition}{note}{Note:}
All work for this lab will be performed exclusively from the Windows
jumphost. No installation or interaction with your local system is
required.
\end{sphinxadmonition}


\section{Lab Topology}
\label{\detokenize{intro:lab-topology}}
The training lab is accessed over remote desktop connection.

Your administrator will provide login credentials and the URL.

Within each lab environment there are the following Virtual Machines:
\begin{itemize}
\item {} 
Windows 7 Jumpbox

\item {} 
Two BIG-IP Virtual Editions (VE) \textendash{} running TMOS 13.0

\item {} 
Two BIG-IQ Virtual Editions (VE) \textendash{} running TMOS 5.2

\item {} 
LAMP Server (Web Servers)

\item {} 
DoSServer

\item {} 
SevOne PLA 2.3.0

\end{itemize}

\sphinxincludegraphics[width=6.47917in,height=3.31250in]{{image3}.png}


\subsection{Lab Components}
\label{\detokenize{intro:lab-components}}
Below are all the IP addresses that will be used during the labs. Please
refer back to this page and use the IP addresses assigned to your site.


\begin{savenotes}\sphinxattablestart
\centering
\begin{tabulary}{\linewidth}[t]{|T|T|}
\hline
&
IP Addresses
\\
\hline
Lampserver
&
10.128.20.150, 10.128.20.160, 10.128.20.170
\\
\hline
\end{tabulary}
\par
\sphinxattableend\end{savenotes}


\chapter{Advanced Firewall Manager, the ENTERPRISE Firewall}
\label{\detokenize{class1/class1:afm-the-enterprise-firewall}}\label{\detokenize{class1/class1::doc}}

\section{Advanced Firewall Manager (AFM) Introduction}
\label{\detokenize{class1/module1/module1:afm-afm-introduction}}\label{\detokenize{class1/module1/module1::doc}}
In this lab, you will create a pool of servers to be load balanced
behind the BIG-IP. You will then experiment with AFM policies to permit
and deny traffic and examine the AFM logging and reporting capabilities.
In subsequent labs you will experiment with the DDoS mitigation features
of AFM.

\begin{sphinxadmonition}{caution}{Caution:}
When you RDP into the Windows box, make sure the time on the windows
client is correct for the current time in the Eastern Time Zone.

If the time needs to be corrected, click on the clock and choose “Change
date and time settings…”

\sphinxincludegraphics[width=3.54122in,height=2.67675in]{{image4}.png}

\sphinxstylestrong{If the time is not corrected please alert an instructor before
proceeding}.
\end{sphinxadmonition}


\subsection{Create a Pool and Virtual Server using REST API}
\label{\detokenize{class1/module1/lab1::doc}}\label{\detokenize{class1/module1/lab1:create-a-pool-and-virtual-server-using-rest-api}}
\sphinxstylestrong{About Representational State Transfer}

Representational State Transfer (REST) describes an architectural style
of web services where clients and servers exchange representations of
resources. The REST model defines a resource as a source of information,
and defines a representation as the data that describes the state of a
resource. REST web services use the HTTP protocol to communicate between
a client and a server, specifically by means of the POST, GET, PUT, and
DELETE methods to create, read, update, and delete elements or
collections. In general terms, REST queries resources for the
configuration objects of a BIG-IP® system, and creates, deletes, or
modifies the representations of those configuration objects. The
iControl® REST implementation follows the REST model by:
\begin{itemize}
\item {} 
Using REST as a resource-based interface, and creating API methods
based on nouns.

\item {} 
Employing a stateless protocol and MIME data types, as well astaking advantage
of the authentication mechanisms and caching built into the HTTP protocol.

\item {} 
Supporting the JSON format for document encoding.

\item {} 
Representing the hierarchy of resources and collections with a Uniform
Resource Identifier (URI) structure.

\item {} 
Returning HTTP response codes to indicate success or failure of an
operation.

\item {} 
Including links in resource references to accommodate discovery.

\end{itemize}

\sphinxstylestrong{About URI format}

The iControl® REST API enables the management of a BIG-IP® device by
using web service requests. A principle of the REST architecture
describes the identification of a resource by means of a Uniform
Resource Identifier (URI). You can specify a URI with a web service
request to create, read, update, or delete some component or module of a
BIG-IP system configuration. In the context of REST architecture, the
system configuration is the representation of a resource. A URI
identifies the name of a web resource; in this case, the URI also
represents the tree structure of modules and components in TMSH.

In iControl REST, the URI structure for all requests includes the string
/mgmt/tm/ to identify the namespace for traffic management. Any
identifiers that follow the endpoint are resource collections.

Tip: Use the default administrative account, admin, for requests to
iControl REST. Once you are familiar with the API, you can create user
accounts for iControl REST users with various permissions.

\sphinxurl{https://management-ip/mgmt/tm/module}

The URI in the previous example designates all of the TMSH subordinate
modules and components in the specified module. iControl REST refers to
this entity as an organizing collection. An organizing collection
contains links to other resources. The management-ip component of the
URI is the fully qualified domain name (FQDN) or IP address of a BIG-IP
device.

Important: iControl REST only supports secure access through HTTPS, so
you must include credentials with each REST call. Use the same
credentials you use for the BIG-IP device manager interface.

For example, use the following URI to access all the components and
subordinate modules in the LTM module:

\sphinxurl{https://192.168.25.42/mgmt/tm/ltm}

The URI in the following example designates all of the subordinate
modules and components in the specified sub-module. iControl REST refers
to this entity as a collection; a collection contains resources.

\sphinxurl{https://management-ip/mgmt/tm/module/sub-module}

The URI in the following example designates the details of the specified
component. The Traffic Management Shell (TMSH) Reference documents the
hierarchy of modules and components, and identifies details of each
component. iControl REST refers to this entity as a resource. A resource
may contain links to sub-collections.

\sphinxhref{https://management-ip/mgmt/tm/module\%5b/sub-module\%5d/component}{https://management-ip/mgmt/tm/module{[}/sub-module{]}/component}

\sphinxstylestrong{About reserved ASCII characters}

To accommodate the BIG-IP® configuration objects that use characters,
which are not part of the unreserved ASCII character set, use a percent
sign (\%) and two hexadecimal digits to represent them in a URI. The
unreserved character set consists of: {[}A - Z{]} {[}a - z{]} {[}0 - 9{]} dash (-),
underscore (\_), period (.), and tilde (\textasciitilde{}).

You must encode any characters that are not part of the unreserved
character set for inclusion in a URI scheme. For example, an IP address
in a non-default route domain that contains a percent sign to indicate
an address in a specific route domain, such as 192.168.25.90\%3, should
be encoded to replace the \%character with \%25.

\sphinxstylestrong{About REST resource identifiers}

A URI is the representation of a resource that consists of a protocol,
an address, and a path structure to identify a resource and optional
query parameters. Because the representation of folder and partition
names in TMSH often includes a forward slash (/), URI encoding of folder
and partition names must use a different character to represent a
forward slash in iControl®

To accommodate the forward slash in a resource name, iControl REST maps
the forward slash to a tilde (\textasciitilde{}) character. When a resource name
includes a forward slash (/) in its name, substitute a tilde (\textasciitilde{}) for the
forward slash in the path. For example, a resource name, such as
/Common/plist1, should be modified to the format shown here:

\sphinxurl{https://management-ip/mgmt/tm/security/firewall/port-list/~Common~plist1}

\sphinxstylestrong{About Postman \textendash{} REST Client}

Postman helps you be more efficient while working with APIs. Postman is
a scratch-your-own-itch project. The need for it arose while one of the
developers was creating an API for his project. After looking around for
a number of tools, nothing felt just right. The primary features added
initially were a history of sent requests and collections. You can find
Postman here: \sphinxhref{http://www.getpostman.com}{www.getpostman.com}

A PostMAN collection has been created to simplify building the pools and
virtuals necessary for the remainder of this course. The collection is
called “Service Provider Specialist Event \textendash{} Lab 1a”. You can
sequentially execute all twelve steps by pressing the send button after
clicking on each call.

\sphinxincludegraphics[width=6.50000in,height=0.55208in]{{image5}.png}

Alternatively, you can run all commands at once using the “Runner”
feature. To use the Runner feature locate the Runner Icon at the top of
POSTMan. Select the appropriate collection and click “Start Test” as
exampled below:

\sphinxincludegraphics[width=6.50000in,height=5.02569in]{{image6}.png}

Once completed, all the necessary nodes, pools, and virtuals for the lab
will have been created. As a general POSTMan rule, you should close the
tabs you’ve opened when you are finished working with them (after each
section). POSTMan has a known bug and will crash when there are too many
tabs opened at once.

Now let’s test the virtual server to ensure it works. On your
workstation open a browser and enter the address of your virtual servers
that you just created (\sphinxhref{http://10.128.10.223}{*http://10.128.10.223*}
and \sphinxhref{http://10.128.10.224}{*http://10.128.10.224*}). Refresh the
browser screen several times (use “\textless{}ctrl\textgreater{}” F5 to ensure you are not
displaying cached objects). Note the \sphinxstylestrong{*Server IP address*} should be
alternating between the three destination servers in your pool
(10.128.20.150, 10.128.20.160, 10.128.20.170). The BIG-IP is load
balancing requests in a round-robin fashion.

Go to bigip1.agility.com (10.0.0.4) and view the statistics for the
\sphinxstylestrong{wildcard\_vs} virtual server and the \sphinxstylestrong{wildcard\_vs\_pool} pool
and its associated members. Go to \sphinxstylestrong{Statistics} \textgreater{} \sphinxstylestrong{Module
Statistics} \textgreater{} \sphinxstylestrong{Local Traffic}. In the \sphinxstylestrong{Statistics Type} drop
down item select \sphinxstylestrong{Pools}.

\sphinxincludegraphics[width=6.49514in,height=1.64583in]{{image7}.png}
\begin{itemize}
\item {} 
You may also go to \sphinxstylestrong{Local Traffic \textgreater{} Pools \textgreater{} Statistics}
\begin{itemize}
\item {} 
Did each pool member receive the same number of connections?

\item {} 
Did each pool member receive approximately the same number of
bytes?

\end{itemize}

\end{itemize}

Try connecting directly to the IP addresses of the servers in the pool
from your browser, and through the virtual server on the BIG-IP and take
note of the Client IP Address on the web page as highlighted below:

\sphinxincludegraphics[width=3.91181in,height=1.96389in]{{image8}.png}
\begin{itemize}
\item {} 
Why does the Source IP address change when going through BIG-IP?
\begin{itemize}
\item {} 
What address is it changing to?

\end{itemize}

\item {} 
Verify that you can connect through the wildcard virtual server using
various ports:
\begin{itemize}
\item {} 
Edit the URL in your browser to
\sphinxhref{http://10.128.10.223:8081}{*http://10.128.10.223:8081*}

\item {} 
Edit the URL in your browser to
\sphinxhref{https://10.128.10.223}{*https://10.128.10.223*}

\item {} 
Edit the URL in your browser to
\sphinxhref{ftp://10.128.10.223}{*ftp://10.128.10.223*}
\begin{itemize}
\item {} 
NOTE: There is no need to login, a prompt will eventually be
displayed.

\end{itemize}

\item {} 
Open Putty (SSH) and access 10.128.10.223
\begin{itemize}
\item {} 
NOTE: you do not need to login, getting a prompt is sufficient
for this test

\end{itemize}

\item {} 
All of these connections should be successful through the BIG-IP,
and should be load balanced to the servers in the pool. Since the
BIG-IP is configured for a wildcard port on the virtual server and
pool, these connections are allowed through.

\item {} 
Close the Web Browsers and Putty.

\end{itemize}

\end{itemize}


\subsection{Advanced Firewall Manager (AFM)}
\label{\detokenize{class1/module1/lab2::doc}}\label{\detokenize{class1/module1/lab2:advanced-firewall-manager-afm}}
Advanced Firewall Manager (AFM) is a new module that was added to TMOS
in version 11.3. F5

BIG-IP Advanced Firewall Manager™ (AFM) is a high-performance ICSA
certified, stateful, full-proxy

network firewall designed to guard data centers against incoming threats
that enter the network on the most widely deployed protocols—including
HTTP/S, SMTP, DNS, SIP, and FTP.

By aligning firewall policies with the applications, they protect,
BIG-IP AFM streamlines application deployment, security, and monitoring.
With its scalability, security, and simplicity, BIG-IP AFM forms the
core of the F5 application delivery firewall solution.

\sphinxincludegraphics[width=6.50000in,height=1.87222in]{{image9}.png}

Some facts below about AFM, and its functionality:
\begin{itemize}
\item {} 
Advanced Firewall Manager (AFM) provides “Shallow” packet inspection
while Application Security Manager (ASM) provides “Deep” packet
inspection. By this we mean that AFM is concerned with source IP
address and port, destination IP address and port, and protocol (this
is also known as 5-tuple/quintuple filtering).

\item {} 
AFM is used to allow/deny a connection before deep packet inspection
ever takes place, think of it as the first line of firewall defense.

\item {} 
AFM is many firewalls in one. You can apply L4 firewall rules to ALL
addresses on the BIG-IP or you can specify BIG-IP configuration
objects (route domains, virtual server, self-IP, and Management-IP).

\item {} 
AFM runs in 2 modes: \sphinxstylestrong{*ADC mode*} and \sphinxstylestrong{*Firewall*} mode. \sphinxstylestrong{*ADC
mode*} is called a “blacklist”, all traffic is allowed to BIG-IP
except traffic that is explicitly DENIED (this is a negative security
model). \sphinxstylestrong{*Firewall mode*} is called a “whitelist”, all traffic is
denied to BIG-IP except traffic that is explicitly ALLOWED. The
latter is typically used when the customer only wants to use us as a
firewall or with LTM.

\item {} 
We are enabling “SERVICE DEFENSE IN DEPTH” versus traditional
“DEFENSE IN DEPTH”. This means, instead of using multiple shallow and
deep packet inspection devices inline increasing infrastructure
complexity and latency, we are offering these capabilities on a
single platform.

\item {} 
AFM is an ACL based firewall. In the old days, we used to firewall
networks using simple packet filters. With a packet filter, if a
packet doesn’t match the filter it is allowed (not good). With AFM,
if a packet does not match criteria, the packet is dropped.

\item {} 
AFM is a stateful packet inspection (SPI) firewall. This means that
BIG-IP is aware of new packets coming to/from BIG-IP, existing
packets, and rogue packets.

\item {} 
AFM adds more than 80 L2-4 denial of service attack vector detections
and mitigations. This may be combined with ASM to provide L4-7
protection.

\item {} 
Application Delivery Firewall is the service defense in depth
layering mentioned earlier. On top of a simple L4 network firewall,
you may add access policy and controls from L4-7 with APM (Access
Policy Manager), or add L7 deep packet inspection with ASM (web
application firewall), You can add DNS DOS mitigation with LTM DNS
Express and GTM + DNSSEC. These modules make up the entire
application delivery firewall (ADF) solution.

\end{itemize}


\subsection{Creating AFM Network Firewall Rules}
\label{\detokenize{class1/module1/lab3::doc}}\label{\detokenize{class1/module1/lab3:creating-afm-network-firewall-rules}}

\subsubsection{Default Actions}
\label{\detokenize{class1/module1/lab3:default-actions}}
The BIG-IP$^{\text{®}}$ Network Firewall provides policy-based access
control to and from address and port pairs, inside and outside of your
network. Using a combination of contexts, the network firewall can apply
rules in many ways, including: at a global level, on a per-virtual
server level, and even for the management port or a self IP address.
Firewall rules can be combined in a firewall policy, which can contain
multiple context and address pairs, and is applied directly to a virtual
server.

By default, the Network Firewall is configured in \sphinxstylestrong{*ADC mode*}, a
default allow configuration, in which all traffic is allowed through the
firewall, and any traffic you want to block must be explicitly
specified.

The system is configured in this mode by default so all traffic on your
system continues to pass after you provision the Advanced Firewall
Manager. You should create appropriate firewall rules to allow necessary
traffic to pass before you switch the Advanced Firewall Manager to
Firewall mode. In \sphinxstylestrong{*Firewall mode*}, a default deny configuration, all
traffic is blocked through the firewall, and any traffic you want to
allow through the firewall must be explicitly specified.

The BIG-IP$^{\text{®}}$ Network Firewall provides policy-based access
control to and from address and port pairs, inside and outside of your
network. By default, the network firewall is configured in ADC mode,
which is a \sphinxstylestrong{*default allow*} configuration, in which all traffic is
allowed to virtual servers and self IPs on the system, and any traffic
you want to block must be explicitly specified. This applies only to the
Virtual Server \& Self IP level on the system.

Important: Even though the system is in a default allow configuration,
if a packet matches no rule in any context on the firewall, a Global
Drop rule drops the traffic.


\subsection{Rule Hierarchy}
\label{\detokenize{class1/module1/lab3:rule-hierarchy}}
With the BIG-IP$^{\text{®}}$ Network Firewall, you use a context to
configure the level of specificity of a firewall rule or policy. For
example, you might make a global context rule to block ICMP ping
messages, and you might make a virtual server context rule to allow only
a specific network to access an application.

Context is processed in this order:
\begin{itemize}
\item {} 
Global

\item {} 
Route domain

\item {} 
Virtual server / self IP

\item {} 
Management port*

\item {} 
Global drop*

\end{itemize}

The firewall processes policies and rules in order, progressing from the
global context, to the route domain context, and then to either the
virtual server or self IP context. Management port rules are processed
separately, and are not processed after previous rules. Rules can be
viewed in one list, and viewed and reorganized separately within each
context. You can enforce a firewall policy on any context except the
management port. You can also stage a firewall policy in any context
except management.

Important: You cannot configure or change the Global Drop context. The
Global Drop context is the final context for traffic. Note that even
though it is a global context, it is not processed first, like the main
global context, but last. If a packet matches no rule in any previous
context, the Global Drop rule drops the traffic.

\sphinxincludegraphics[width=4.74410in,height=5.21054in]{{image10}.png}


\subsection{Create and View Log Entries}
\label{\detokenize{class1/module1/lab3:create-and-view-log-entries}}
In this section, you will generate various types of traffic through the
virtual server as you did previously, but now you will view the log
entries using the network firewall log. Open the \sphinxstylestrong{*Security \textgreater{} Event
Logs \textgreater{} Network \textgreater{} Firewall*} page on bigip01.agility.com (10.0.0.4). The
log file is empty because no traffic has been sent to the virtual server
since you enabled logging.
\begin{itemize}
\item {} 
Open a new Web browser and access your \sphinxstylestrong{*wildcard\_vs*}
\sphinxhref{http://10.128.10.223}{*http://10.128.10.223*}

\item {} 
Edit the URL to
\sphinxhref{http://10.128.10.223:8081}{*http://10.128.10.223:8081*}

\item {} 
Edit the URL to \sphinxhref{https://10.128.10.223}{*https://10.128.10.223*}

\item {} 
Open either Chrome or Firefox and access
\sphinxhref{ftp://10.128.10.223}{*ftp://10.128.10.223*}

\item {} 
Open Putty and access 10.128.10.223, you do not need to log in.

\item {} 
Close all Web browsers and Putty sessions.

\item {} 
In the Configuration Utility, reload the Firewall log page.
\begin{itemize}
\item {} 
\sphinxstylestrong{*Security \textgreater{} Event Logs \textgreater{} Network \textgreater{} Firewall*}

\end{itemize}

\item {} 
Sort the list in descending order by the Time column.

\end{itemize}

\sphinxincludegraphics[width=7.25069in,height=1.99792in]{{image11}.png}

Examine the Source Address and Destination Port values. Note how
requests for all services were established and none were blocked.
Although we will not configure external logging in this lab, you should
be aware that the BIG-IP supports high speed external logging in various
formats including \sphinxstylestrong{*SevOne*}, \sphinxstylestrong{*Splunk*} and \sphinxstylestrong{*ArcSight*}. Below
are some examples of AFM Firewall and DoS logs being presented by
SevOne:

\sphinxincludegraphics[width=6.50000in,height=1.86458in]{{image12}.png}


\subsection{Create a Rule List}
\label{\detokenize{class1/module1/lab3:create-a-rule-list}}
Rule lists are a way to group a set of individual rules together and
apply them to the active rule base as a group. A typical use of a rule
list would be for a set of applications that have common requirements
for access protocols and ports. As an example, most web applications
would require TCP port 80 for HTTP and TCP port 443 for SSL/TLS. You
could create a Rule list with these protocols, and apply them to each of
your virtual servers.

Let’s examine some of the default rule lists that are included with AFM.
Go to \sphinxstylestrong{*Security \textgreater{}Network Firewall \textgreater{} Rule Lists*}. They are:
\begin{itemize}
\item {} 
\_sys\_self\_allow\_all

\item {} 
\_sys\_self\_allow\_defaults

\item {} 
\_sys\_self\_allow\_management

\end{itemize}

\sphinxincludegraphics[width=6.50000in,height=1.46319in]{{image13}.png}

If you click on \sphinxstylestrong{*\_sys\_self\_allow\_management*} you’ll see that it
is made up of two different rules that will allow management traffic
(port 22/SSH and port 443 HTTPS). Instead of applying multiple rules
over and over across multiple virtual servers, you can put them in a
rule list and then apply the rule list as an ACL.

\sphinxincludegraphics[width=6.50000in,height=0.80278in]{{image14}.png}

On bigip01.agility.com (10.0.0.4) create a rule list to allow Web
traffic. A logical container must be created before the individual rules
can be added. You will create a list with three rules, to allow port 80
(HTTP), allow port 443 (HTTPS) and reject traffic from a specific IP
subnet. First you need to create a container for the rules by going to
\sphinxstylestrong{*Security \textgreater{} Network Firewall \textgreater{} Rule Lists*} and select \sphinxstylestrong{*Create*.}
For the \sphinxstylestrong{*Name*} enter \sphinxstylestrong{*web\_rule\_list*}, provide an optional
description and then click \sphinxstylestrong{*Finished*.}

\sphinxincludegraphics[width=3.26105in,height=1.47052in]{{image15}.png}

Edit the \sphinxstylestrong{*web\_rule\_list*} by selecting it in the Rule Lists table,
then click the \sphinxstylestrong{*Add*} button in the Rules section. Here you will add
three rules into the list; the first is a rule to allow HTTP.

\sphinxincludegraphics[width=6.30515in,height=1.66925in]{{image16}.png}


\begin{savenotes}\sphinxattablestart
\centering
\begin{tabulary}{\linewidth}[t]{|T|T|}
\hline

\sphinxstylestrong{Name}
&
allow\_http
\\
\hline
\sphinxstylestrong{Protocol}
&
TCP
\\
\hline
\sphinxstylestrong{Source}
&
Leave at Default of \sphinxstylestrong{*Any*}
\\
\hline
\sphinxstylestrong{Destination Address}
&
Any
\\
\hline
\sphinxstylestrong{Destination Port}
&
Specify Single Port \sphinxstylestrong{*80*}, then click \sphinxstylestrong{*Add*}
\\
\hline
\sphinxstylestrong{Action}
&
Accept
\\
\hline
\sphinxstylestrong{Logging}
&
Enabled
\\
\hline
\end{tabulary}
\par
\sphinxattableend\end{savenotes}

\sphinxincludegraphics[width=3.34926in,height=3.60993in]{{image17}.png}

Select \sphinxstylestrong{Repeat} when done.

Create a rule to allow HTTPS.


\begin{savenotes}\sphinxattablestart
\centering
\begin{tabulary}{\linewidth}[t]{|T|T|}
\hline

\sphinxstylestrong{Name}
&
allow\_https
\\
\hline
\sphinxstylestrong{Protocol}
&
TCP
\\
\hline
\sphinxstylestrong{Source}
&
Leave at Default of Any
\\
\hline
\sphinxstylestrong{Destination Address}
&
Any
\\
\hline
\sphinxstylestrong{Destination Port}
&
Specify Single Port 443, then click \sphinxstylestrong{*Add*}
\\
\hline
\sphinxstylestrong{Action}
&
Accept
\\
\hline
\sphinxstylestrong{Logging}
&
Enabled
\\
\hline
\end{tabulary}
\par
\sphinxattableend\end{savenotes}

Select \sphinxstylestrong{Finished} when done. Create another rule by clicking \sphinxstylestrong{*Add*}
to reject all access from the 10.0.10.0/24 network.


\begin{savenotes}\sphinxattablestart
\centering
\begin{tabulary}{\linewidth}[t]{|T|T|}
\hline

\sphinxstylestrong{Name}
&
reject\_10\_0\_10\_0
\\
\hline
\sphinxstylestrong{Protocol}
&
Any
\\
\hline
\sphinxstylestrong{Source}
&
Address 10.0.10.0/24, then click \sphinxstylestrong{*Add*}
\\
\hline
\sphinxstylestrong{Destination Address}
&
Any
\\
\hline
\sphinxstylestrong{Destination Port}
&
Any
\\
\hline
\sphinxstylestrong{Action}
&
Reject
\\
\hline
\sphinxstylestrong{Logging}
&
Enabled
\\
\hline
\end{tabulary}
\par
\sphinxattableend\end{savenotes}

Select \sphinxstylestrong{*Finished*} when done. When you exit you’ll notice the reject
rule is after the \sphinxstylestrong{*allow\_http*} and \sphinxstylestrong{*allow\_https*} rules. This
means that HTTP and HTTPS traffic from 10.0.10.0/24 will be accepted,
while all other traffic from this subnet will be rejected based on the
ordering of the rules as seen below:

\sphinxincludegraphics[width=6.50000in,height=0.48681in]{{image18}.png}


\subsection{Create a Policy with a Rule List}
\label{\detokenize{class1/module1/lab3:create-a-policy-with-a-rule-list}}
Policies are a way to group a set of individual rules together and apply
them to the active policy base as a group. A typical use of a policy
list would be for a set of rule lists that have common requirements for
access protocols and ports.

Create a policy list to allow the traffic you created in the rule list
in the previous section. A logical container must be created before the
individual rules can be added. First you need to create a container for
the policy by going to \sphinxstylestrong{*Security \textgreater{} Network Firewall \textgreater{} Policies*} and
select \sphinxstylestrong{*Create*.} For the \sphinxstylestrong{*Name*} enter \sphinxstylestrong{*rd\_0\_policy*},
provide an optional description and then click \sphinxstylestrong{*Finished*.}

\sphinxincludegraphics[width=4.92847in,height=1.35694in]{{image19}.png}

Edit the \sphinxstylestrong{*rd\_0\_policy*} by selecting it in the Policy Lists table,
then click the \sphinxstylestrong{*Add*} button in the Rules section. Here you will add
the rule list you created in the previous section. For the \sphinxstylestrong{*Name*}
enter \sphinxstylestrong{*web\_policy*}, provide an optional description, for type
select \sphinxstylestrong{*Rule List,*} select the Rule List “\sphinxstylestrong{*web\_rule\_list*}”
and then click \sphinxstylestrong{*Finished*.}

\sphinxincludegraphics[width=4.04722in,height=1.93264in]{{image20}.png}

When finished your policy should look similar to the screenshot below.

\sphinxincludegraphics[width=6.47361in,height=1.78958in]{{image21}.png}


\subsection{Add the Rule List to a Route Domain}
\label{\detokenize{class1/module1/lab3:add-the-rule-list-to-a-route-domain}}
In this section, you are going to attach the rule to a route domain
using the \sphinxstylestrong{*Security*} selection in the top bar within the \sphinxstylestrong{*Route
Domain*} GUI interface. Go to \sphinxstylestrong{*Network*}, then click on \sphinxstylestrong{*Route
Domains*}, then select the hyperlink for route domain \sphinxstylestrong{*0*}. Now
click on the \sphinxstylestrong{*Security*} top bar selection, which is a new option
that was added in version 11.3. From the Network Firewall Enforcement
dropdown menu select enabled. Select the policy you just created
“rd\_0\_policy” and click update.

Review the rules that are now applied to this route domain.

\sphinxincludegraphics[width=6.49514in,height=2.50556in]{{image22}.png}

We will insert a reject clause into the existing rule list so that you
can examine different types of log entries. Go to \sphinxstylestrong{*Security \textgreater{} Network
Firewall \textgreater{} Rule Lists*}. Select the \sphinxstylestrong{*web\_rule\_list*} you created
earlier so that you may edit it, and then click the \sphinxstylestrong{*Add*} button.

\sphinxincludegraphics[width=6.50000in,height=0.78750in]{{image23}.png}

For \sphinxstylestrong{*Name*} configure \sphinxstylestrong{*reject\_all*}, and leave all options
default except set \sphinxstylestrong{*Action*} for \sphinxstylestrong{*Reject*}, and set \sphinxstylestrong{*Logging*}
to \sphinxstylestrong{*Enabled*}, then click \sphinxstylestrong{*Finished*}.

\sphinxincludegraphics[width=2.49327in,height=2.28456in]{{image24}.png}

Your rule set should look similar to the screenshot below:

\sphinxincludegraphics[width=6.50000in,height=0.91667in]{{image25}.png}


\subsection{Creating Rules and Policy via REST API}
\label{\detokenize{class1/module1/lab4:creating-rules-and-policy-via-rest-api}}\label{\detokenize{class1/module1/lab4::doc}}
The RESTful API is also capable of AFM modifications. To add the same
rules and policy to bigip02.agility.com (10.0.0.5), simply follow the
calls in the collection “Service Provider Specialist Event - Lab 1b”.
These calls can be run individually in the sequence provided or using
Runner as exampled below:

\sphinxincludegraphics[width=6.50000in,height=5.02569in]{{image26}.png}


\subsection{Test the New Firewall Rules}
\label{\detokenize{class1/module1/lab4:test-the-new-firewall-rules}}
Once again you will generate traffic through the BIG-IP VE system using
the \sphinxstylestrong{*wildcard\_vs*} virtual server and then

view the AFM (firewall) logs.
\begin{itemize}
\item {} 
Open a new Web browser and access
\sphinxhref{https://10.128.10.223}{*https://10.128.10.223*}

\item {} 
Edit the URL to \sphinxhref{http://10.128.10.223}{*http://10.128.10.223*}

\item {} 
Edit the URL to
\sphinxhref{http://10.128.10.223:8081}{*http://10.128.10.223:8081*}

\item {} 
Open either Chrome or Firefox and access
\sphinxhref{ftp://10.128.10.223}{*ftp://10.128.10.223*}

\item {} 
Open Putty and access 10.128.10.223

\end{itemize}

In the Configuration Utility, open the \sphinxstylestrong{*Security \textgreater{} Event Logs \textgreater{}
Network \textgreater{} Firewall*} page.

Access for port 80 was granted to a host using the web\_rule\_list:
\sphinxstylestrong{*allow\_http*} rule and access for 443 was granted using the
web\_rule\_list: \sphinxstylestrong{*allow\_https rule*}.

\sphinxincludegraphics[width=6.50000in,height=1.46111in]{{image27}.png}

Requests for port 8081, 21, and 22 were all rejected due to the
reject\_all rule.

\sphinxincludegraphics[width=6.50000in,height=1.71319in]{{image28}.png}

You may verify this, by going to \sphinxstylestrong{*Network \textgreater{} Route Domains*}, then
selecting the hyperlink for route domain 0, then select \sphinxstylestrong{*Security*}.
Note the \sphinxstylestrong{*Count*} field next to each rule as seen below. Also note
how each rule will also provide a \sphinxstylestrong{*Latest Matched*} field so you will
know the last time each rule was hit:

\sphinxincludegraphics[width=6.50000in,height=1.50000in]{{image29}.png}

If you wanted to reject all traffic from the 10.0.10.0/24 network
including HTTP \& HTTPS, then open the \sphinxstylestrong{*Security \textgreater{} Network Firewall \textgreater{}
Rules Lists*} page. Select \sphinxstylestrong{*web\_rule\_list*} and click the
\sphinxstylestrong{*Reorder*} button. Use your mouse to move the
\sphinxstylestrong{*reject\_10\_0\_10\_0*} rule above the \sphinxstylestrong{*allow\_http rule*}.

\sphinxincludegraphics[width=6.50000in,height=0.88958in]{{image30}.png}

Click \sphinxstylestrong{*Update*}.


\subsection{Creating a Rule List for Multiple Services}
\label{\detokenize{class1/module1/lab5::doc}}\label{\detokenize{class1/module1/lab5:creating-a-rule-list-for-multiple-services}}
Rules and Rule Lists can also be created and attached to a context from
the Active Rules section of the GUI. Go to the \sphinxstylestrong{*Security \textgreater{} Network
Firewall \textgreater{} Rule Lists*} and create a \sphinxstylestrong{*Rule List*} called
\sphinxstylestrong{*common\_services\_rule\_list*} then click \sphinxstylestrong{*Finished*}. Enter the
rule list by clicking on its hyperlink, then in the \sphinxstylestrong{*Rules*} section
click \sphinxstylestrong{*Add*}, and add the following information:

\sphinxincludegraphics[width=3.40998in,height=1.25178in]{{image31}.png}

Add another rule using the following information:

\sphinxincludegraphics[width=3.40031in,height=1.18502in]{{image32}.png}

\sphinxincludegraphics[width=6.50000in,height=0.66806in]{{image33}.png}


\subsection{Add Another Rule List to the Policy}
\label{\detokenize{class1/module1/lab5:add-another-rule-list-to-the-policy}}
Use the \sphinxstylestrong{*Policies*} page to add the new firewall rule list to the
\sphinxstylestrong{*rd\_0\_policy*}. Open the \sphinxstylestrong{*Security \textgreater{} Network Firewall \textgreater{}
Policies*} page. Click on the policy name to modify the policy.

The only current active rule list is for the web\_policy. Click on
\sphinxstylestrong{*Add*} to add the new rule list you just created.

Configure as seen below, for \sphinxstylestrong{*Name*} use
\sphinxstylestrong{*allow\_common\_services*}, for \sphinxstylestrong{*Order*} select \sphinxstylestrong{*Before*}
\sphinxstylestrong{*web\_policy*}, and for \sphinxstylestrong{*Type*} select \sphinxstylestrong{*Rule List*} and select
the rule \sphinxstylestrong{*common\_services\_rule\_list*}, then click \sphinxstylestrong{*Finished*}.

\sphinxincludegraphics[width=3.43330in,height=1.36973in]{{image34}.png}

You should see the policy similar to the one below:

\sphinxincludegraphics[width=6.50000in,height=1.08264in]{{image35}.png}

At this point all FTP and SSH traffic will be allowed, before BIG-IP AFM
reaches the second rule list.


\subsection{Test Access to the Wildcard Virtual Server}
\label{\detokenize{class1/module1/lab5:test-access-to-the-wildcard-virtual-server}}\begin{itemize}
\item {} 
Open a new Web browser and access
\sphinxhref{http://10.128.10.223:8081}{*http://10.128.10.223:8081*}

\item {} 
Edit the URL to \sphinxhref{https://10.128.10.223}{*https://10.128.10.223*}

\item {} 
Edit the URL to \sphinxhref{http://10.128.10.223}{*http://10.128.10.223*}

\item {} 
Open either Chrome or Firefox and access
\sphinxhref{ftp://10.128.10.223}{*ftp://10.128.10.223*}

\item {} 
Open Putty and access 10.128.10.223

\item {} 
Close all Web browsers and Putty sessions.

\end{itemize}

You should notice HTTP, HTTPS, FTP, and SSH traffic is now allowed
through the firewall, while traffic destined to port 8081 is still
rejected. If you do not see the 8081 request failing, you may need to
refresh to avoid using the browser cache.

Next, you’ll see how easy it is to search through the logs. In the
Configuration Utility, open the \sphinxstylestrong{*Security \textgreater{} Event Logs \textgreater{} Network \textgreater{}
Firewall*} page. Click \sphinxstylestrong{*Custom Search*}. Select a \sphinxstylestrong{*Reject*} entry
in the list (just the actual word “reject”) and drag it to the custom
search area.

\sphinxincludegraphics[width=3.29252in,height=1.23036in]{{image36}.png}

Click \sphinxstylestrong{*Search*}. This will filter the logs so that it just displays
all rejected entries.


\subsection{View Firewall Reports}
\label{\detokenize{class1/module1/lab6:view-firewall-reports}}\label{\detokenize{class1/module1/lab6::doc}}
View several of the built-in network firewall reports and graphs on the
BIG-IP system. On BIG-IP01 (10.0.0.4) open the \sphinxstylestrong{*Security \textgreater{}Reporting \textgreater{}
Network \textgreater{} Enforced Rules*} page. The default report shows all the rule
contexts that were matched in the past hour. The default view gives
reports per Context, in the drop-down menu select \sphinxstylestrong{*Rules
(Enforced)*}.

\sphinxincludegraphics[width=6.49514in,height=3.31250in]{{image37}.png}

From the \sphinxstylestrong{*View By*} list, select \sphinxstylestrong{*Destination Ports (Enforced)*}.

\sphinxincludegraphics[width=2.65075in,height=1.53934in]{{image38}.png}

This redraws the graph to report more detail for all of the destination
ports that matched an ACL.

\sphinxincludegraphics[width=6.47361in,height=3.42083in]{{image39}.png}

From the \sphinxstylestrong{*View By*} list, select \sphinxstylestrong{*Source IP Addresses
(Enforced)*}. This shows how source IP addresses matched an ACL clause:

\sphinxincludegraphics[width=6.47361in,height=3.21042in]{{image40}.png}


\subsection{AFM Reference Material}
\label{\detokenize{class1/module1/lab6:afm-reference-material}}\begin{itemize}
\item {} 
\sphinxhref{http://www.networkworld.com/reviews/2013/072213-firewall-test-271877.html}{Network World Review of AFM: F5 data center firewall aces performance test}

\item {} 
\sphinxhref{http://www.f5.com/products/big-ip/big-ip-advanced-firewall-manager/overview}{AFM Product Details}

\item {} 
\sphinxhref{https://support.f5.com/content/kb/en-us/products/big-ip-afm/manuals/product/f5-afm-operations-guide/\_jcr\_content/pdfAttach/download/file.res/f5-afm-operations-guide.pdf}{AFM Operations Guide}

\end{itemize}


\section{AFM Packet Tester}
\label{\detokenize{class1/module2/module2:afm-packet-tester}}\label{\detokenize{class1/module2/module2::doc}}
New in the v13 release of the BIG-IP Advanced Firewall Manager is the capability to insert a packet trace into the internal flow so you can analyze what component within the system is allowing or blocking packets based on your configuration of features and rule sets.

\sphinxincludegraphics[width=6.50000in,height=3.44792in]{{image42}.png}

The packet tracing is inserted at L3 immediately prior to the Global IP
intelligence. Because it is after the L2 section, this means that a) we
cannot capture in tcpdump so we can’t see them in flight and b) no
physical layer details will matter as it relates to testing. That said,
it’s incredibly useful for what is and is not allowing your packets
through. You can insert tcp, udp, sctp, and icmp packets, with a limited
set of (appropriate to each protocol) attributes for each.


\subsection{Create and View Packet Tracer Entries}
\label{\detokenize{class1/module2/lab1:create-and-view-packet-tracer-entries}}\label{\detokenize{class1/module2/lab1::doc}}
In this section, you will generate various types of traffic as you did
previously, but now you will view the flow using the network packet
tracer. Login to bigip01.agility.com (10.0.0.4), open the \sphinxstylestrong{*Network \textgreater{}
Network Security \textgreater{} Packet Tester*} page.

\sphinxincludegraphics[width=6.48958in,height=3.44792in]{{image43}.png}

Create a packet test with the following parameters.


\begin{savenotes}\sphinxattablestart
\centering
\begin{tabulary}{\linewidth}[t]{|T|T|}
\hline

\sphinxstylestrong{Protocol}
&
TCP
\\
\hline
\sphinxstylestrong{TCP Flags}
&
SYN
\\
\hline
\sphinxstylestrong{Source}
&
IP - 1.2.3.4
Port \textendash{} 9999
Vlan \textendash{} External
\\
\hline
\sphinxstylestrong{TTL}
&
255
\\
\hline
\sphinxstylestrong{Destination}
&
IP \textendash{} 10.128.10.223
Port \textendash{} 80
\\
\hline
\sphinxstylestrong{Trace Options}
&
Use Staged Policy \textendash{} no
Trigger Log - no
\\
\hline
\end{tabulary}
\par
\sphinxattableend\end{savenotes}

Click Run Trace to view the response. Your output should resemble the
allowed flow as shown below:

\sphinxincludegraphics[width=6.48958in,height=2.60417in]{{image44}.png}

Click \sphinxstylestrong{New Packet Trace} (optionally do not clear the existing data).

Create a packet test with the following parameters.


\begin{savenotes}\sphinxattablestart
\centering
\begin{tabulary}{\linewidth}[t]{|T|T|}
\hline

\sphinxstylestrong{Protocol}
&
TCP
\\
\hline
\sphinxstylestrong{TCP Flags}
&
SYN
\\
\hline
\sphinxstylestrong{Source}
&
IP - 1.2.3.4
Port \textendash{} 9999
Vlan \textendash{} External
\\
\hline
\sphinxstylestrong{TTL}
&
255
\\
\hline
\sphinxstylestrong{Destination}
&
IP \textendash{} 10.128.10.223
Port - \sphinxstylestrong{8081}
\\
\hline
\sphinxstylestrong{Trace Options}
&
Use Staged Policy \textendash{} no
Trigger Log - yes
\\
\hline
\end{tabulary}
\par
\sphinxattableend\end{savenotes}

Click Run Trace to view the response. Your output should resemble the
allowed flow as shown below:

\sphinxincludegraphics[width=6.49514in,height=3.71389in]{{image45}.png}

You can click on the /common/rd\_0\_policy hyperlink to examine the
policy which is rejecting the request.

You can also perform the same tests using the API. To do, so launch
POSTMan and use the collections for Lab 2. The first call will mirror
what was sent in the accept. The second call will mirror what was sent
in the rejected response. Example shown below:

\sphinxincludegraphics[width=6.48958in,height=1.61458in]{{image46}.png}

If you examine the JSON output for the second request, the rejected
request, you’ll notice the following lines within the JSON output:

\begin{sphinxVerbatim}[commandchars=\\\{\}]
\PYG{p}{\PYGZob{}}

\PYG{n+nt}{\PYGZdq{}acl\PYGZus{}rtdom\PYGZus{}policy\PYGZus{}type\PYGZdq{}}\PYG{p}{:} \PYG{p}{\PYGZob{}}
             \PYG{n+nt}{\PYGZdq{}description\PYGZdq{}}\PYG{p}{:} \PYG{l+s+s2}{\PYGZdq{}Enforced\PYGZdq{}}
\PYG{p}{\PYGZcb{}}\PYG{p}{,}
\PYG{n+nt}{\PYGZdq{}acl\PYGZus{}rtdom\PYGZus{}rule\PYGZus{}name\PYGZdq{}}\PYG{p}{:} \PYG{p}{\PYGZob{}}
             \PYG{n+nt}{\PYGZdq{}description\PYGZdq{}}\PYG{p}{:} \PYG{l+s+s2}{\PYGZdq{}/Common/web\PYGZus{}rule\PYGZus{}list:reject\PYGZus{}all\PYGZdq{}}
\PYG{p}{\PYGZcb{}}
\end{sphinxVerbatim}

This is the same rule which was show in the UI packet tester for the
rule that is not permitting this request. If you search for the keys
above in the permitted flow you’ll notice the output is quite different.

These are possible values:

\sphinxcode{var aclActionType = \{"0":"Drop","1":"Reject","2":"Allow","3":"Decisive
Allow","4":"Default","5":"Prior Decisive","6":"Default Rule
Allow","7":"Default Rule Drop","8":"Default Rule Reject","9":"Allow (No
Policy)", "10":"Allow (No Match)","11":"Prior Drop","12":"Not
Applicable","13":"Drop (Flow Miss)","14":"Prior Reject"\}}

\sphinxcode{var dosActionType = \{"0":"Default","1":"Allow (No Anomaly)","2":"White
List","3":"Allow (Anomaly)","4":"Drop (Rate Limited)","5":"Drop
(Attack)", "6":"Prior White List","7":"Allow (No Policy)","8":"Prior
Drop","9":"Drop (Flow Miss)","10":"Not Applicable","11":"Prior Reject"\}}

\sphinxcode{var ipiActionType = \{"0":"Default","1":"Allow","2":"Drop","3":"Allow
(White List)","4":"Allow (No Policy)", "5":"Allow (No Match)","6":"Prior
Drop","7":"Drop (Flow Miss)","8":"Not Applicable","9":"Prior Reject"\};}

acl\_device\_is\_default\_rule = could be true or false.


\section{DDoS Protection with AFM}
\label{\detokenize{class1/module3/module3::doc}}\label{\detokenize{class1/module3/module3:ddos-protection-with-afm}}
During this lab, you will configure the BIG-IP system to detect and
report on various network level Denial of Service events. You will then
run simulated attacks against the BIG-IP and verify the mitigation,
reporting and logging of these attacks.


\subsection{Create a Pool and Virtual Server using REST API}
\label{\detokenize{class1/module3/lab1::doc}}\label{\detokenize{class1/module3/lab1:create-a-pool-and-virtual-server-using-rest-api}}
Use the POSTMan collection named “Service Provider Specialist Event -
Lab 3a” to create the necessary pools and virtual servers for this
exercise.

Verify connectivity to the new virtual server by opening a browser and
connecting to \sphinxhref{http://10.128.10.20}{*http://10.128.10.20*}


\subsection{Configuring DoS Protection}
\label{\detokenize{class1/module3/lab1:configuring-dos-protection}}
Since we are in a lab environment with no production traffic, we will
need to lower some of the default values for DoS detection values so
that attacks are seen in a timely manner. We are also going to verify
the DoS events are logged locally. Log into bigip01.agility.com
(10.0.0.4), access the \sphinxstylestrong{*Security \textgreater{} DoS Protection \textgreater{} Device*}
\sphinxstylestrong{*Configuration\(\rightarrow\) Properties*} page. From the \sphinxstylestrong{*Log Publisher*} list,
verify \sphinxstylestrong{*local-db-publisher*} is selected.

\sphinxincludegraphics[width=6.50000in,height=2.14583in]{{image47}.png}
\begin{itemize}
\item {} 
Select Network Security from the Device Configuration drop down at
the top

\end{itemize}

\sphinxincludegraphics[width=6.50000in,height=3.44792in]{{image48}.png}
\begin{itemize}
\item {} 
Select the + sign next to \sphinxstylestrong{*Bad-Header \textendash{} Ipv4*}

\item {} 
Then select \sphinxstylestrong{*Bad IP TTL Value*}.

\item {} 
Specify the following threshold values and Click \sphinxstylestrong{*Update*} when
finished:

\end{itemize}


\begin{savenotes}\sphinxattablestart
\centering
\begin{tabulary}{\linewidth}[t]{|T|T|}
\hline

Detection Threshold PPS
&
25
\\
\hline
Detection Threshold Percent
&
100
\\
\hline
Leak Limit/Rate Limit PPS
&
25
\\
\hline
\end{tabulary}
\par
\sphinxattableend\end{savenotes}

This is what set’s F5’s BIG-IP apart from other offerings. It monitors
for DoS activity and when a DoS event is detected, it will not block all
traffic from a IP address, as this could affect legitimate traffic such
as that behind a proxy. Instead, the Big-IP will limit only the
offending traffic allowing legitimate traffic to pass through. Below is
some background on how the detection mechanism works:

\sphinxstylestrong{Detection Threshold PPS}: This is the number of packets per
second (of this attack type) that the BIG-IP system uses to
determine if an attack is occurring. When the number of packets per
second goes above the threshold amount, the BIG-IP system logs and
reports the attack, and then continues to check every second, and
marks the threshold as an attack if the threshold is exceeded. The
default value is 10,000 packets per second, but we’ll change the
values to 25 packets per second for the purposes of this demo.

\sphinxstylestrong{Detection Threshold Percent}: This is the percentage increase
value that specifies an attack is occurring. The BIG-IP system
compares the current rate to an average rate from the last hour. For
example, if the average rate for the last hour is 1000 packets per
second, and you set the percentage increase threshold to 100, an
attack is detected at 100 percent above the average, or 2000 packets
per second. When the threshold is passed, an attack is logged and
reported.

The BIG-IP system then automatically institutes a rate limit equal
to the average for the last hour, and all packets above that limit
are dropped. The BIG-IP system continues to check every second until
the incoming packet rate drops below the percentage increase
threshold. Rate limiting continues until the rate drops below the
specified limit again. The default value is 500 percent, but we’ll
change the values to 100 percent for the purposes of this demo. This
is the lowest value allowed for this setting.

\sphinxstylestrong{Rate/Leak Limit}: This is the value, in packets per second that
cannot be exceeded by packets of this type. All packets of this type
over the threshold are dropped. Rate limiting continues until the
rate drops below the specified limit again. The default value is
10,000 packets per second, but we’ll change the values to 25 packets
per second.

We will set the thresholds for other DDoS events, but rather than go
through the GUI for each one, we will set the thresholds for all of them
at once using tmsh CLI commands. To do so go to the following URL to see
the tmsh commands that will be used:

\sphinxhref{http://10.128.20.150/ddos-commands.txt}{*http://10.128.20.150/ddos-commands.txt*}

Optionally, you can use the POSTMan collection “Service Provider
Specialist Event - Lab 3b” to modify the values on both devices.

Copy all the DDoS commands and then open up an SSH session (via Putty or
similar program) to the management IP address of bigip01.agility.com
(10.0.0.4). Login with the following credentials:
\begin{itemize}
\item {} 
User: root

\item {} 
Password: 401elliottW!

\end{itemize}

Once you are connected paste in the tmsh commands from the web page into
the SSH session to set the DoS thresholds. The following parameters will
be set:
\begin{itemize}
\item {} 
\sphinxstylestrong{Bad Header \textendash{} IPv4}

\item {} 
Bad IP Version

\item {} 
Header Length \textgreater{} L2 Length

\item {} 
Header Length Too Short

\item {} 
IP Error Checksum

\item {} 
IP Length \textgreater{} L2 Length

\item {} 
IP Source Address == Destination Address

\item {} 
L2 length \textgreater{}\textgreater{} IP length

\item {} 
No L4

\item {} 
\sphinxstylestrong{Bad Headers - TCP}

\item {} 
Bad TCP Checksum

\item {} 
Bad TCP Flags (All Flags Set)

\item {} 
FIN Only Set

\item {} 
SYN \& FIN Set

\item {} 
TCP Header Length \textgreater{} L2 Length

\item {} 
TCP Header Length Too Short (Length \textless{} 5)

\item {} 
\sphinxstylestrong{Flood}

\item {} 
ICMP Flood

\item {} 
\sphinxstylestrong{Fragmentation}

\item {} 
IP Fragment

\end{itemize}

Close the PuTTY session to disconnect from the BIG-IP.


\subsection{Run an Attack Script against your Virtual Server}
\label{\detokenize{class1/module3/lab2:run-an-attack-script-against-your-virtual-server}}\label{\detokenize{class1/module3/lab2::doc}}
Open an SSH session to the Ubuntu server \sphinxstylestrong{10.128.10.250} using Putty
or the command line. Login using the username \sphinxstylestrong{root} with the password
\sphinxstylestrong{default}.

Once logged in, list the contents of the current directory using the
\sphinxstylestrong{*ls*} command. You should see a filename similar to
\sphinxstylestrong{*dos-attack-2xx-commands.txt*} file. This file contains various DoS
attack commands that you will run from the Ubuntu machine you are
currently logged into. It may be easiest to copy the file or its
contents to your local desktop or open another SSH session so you will
have easy access to the commands while you open a program on the Ubuntu
server called \sphinxstylestrong{*Scapy*} to run the DDoS commands.

Scapy is a powerful interactive packet manipulation program. It is able
to forge or decode packets of a wide number of protocols, send them on
the wire, capture them, match requests and replies, and much more. It
can easily handle most classical tasks like scanning, tracerouting,
probing, unit tests, attacks or network discovery (it can replace hping,
85\% of nmap, arpspoof, arp-sk, arping, tcpdump, tethereal, p0f, etc.).
If you want to learn more about Scapy the link below is provided for
reference:

\sphinxhref{http://www.secdev.org/projects/scapy/}{*http://www.secdev.org/projects/scapy/*}

We will be using Scapy to create specific attacks to launch at your
Virtual Server. We’ll then verify the logging and reporting as well as
attack mitigation of the BIG-IP.

While logged into the Ubuntu server type the following command:
\sphinxstylestrong{*scapy*}

Copy the first attack command for \sphinxstylestrong{*Bad IP TTL value*}, and then paste
the command in the scapy terminal window and hit enter. You should see
dots move across the screen indicating that the attack is being sent.

\sphinxincludegraphics[width=3.62599in,height=2.28716in]{{image49}.png}
\begin{itemize}
\item {} 
This attack will launch 4000 packets that are configured to send IP
requests with a TTL value of 0

\end{itemize}


\subsection{View DoS Logging}
\label{\detokenize{class1/module3/lab3:view-dos-logging}}\label{\detokenize{class1/module3/lab3::doc}}
Use the BIG-IP configuration utility to view the DoS logging. While the
attacks are running, access \sphinxstylestrong{Security \textgreater{} DoS Protection \textgreater{} DoS
Overview}.

\sphinxincludegraphics[width=6.50000in,height=1.60000in]{{image50}.png}

Note that the attack vector properties are available for modification to
the right. This is useful during an attack if a value needs to be
immediately modified.

In the configuration utility access, the \sphinxstylestrong{*Security \textgreater{} Event Logs \textgreater{} DoS
\textgreater{} Network \textgreater{} Events*} page. If necessary, sort the list in descending
order by the Time column.

\sphinxincludegraphics[width=6.50000in,height=0.84514in]{{image51}.png}

There should be an entry that was created when the BIG-IP identified the
start of the DoS attack, and then one or more entries for dropped
packets every second that the DoS attack continued. Eventually there
will be an entry for when the BIG-IP determines that the DoS attack has
stopped.

\sphinxincludegraphics[width=6.50000in,height=1.45486in]{{image52}.png}

Note the \sphinxstylestrong{*Action*} and \sphinxstylestrong{*Dropped Packets*} column, this indicates
that the BIG-IP not only detected the attack, but it also mitigated the
attack by dropping the packets with the bad IP TTL value.

Repeat the same steps for all the network attacks in the file and be
sure to verify the DoS logs and ensure each event has a start and a
stop.

Once you are finished running and verifying all the attacks, we will
then examine the network DoS reporting capabilities within the BIG-IP.
In the configuration utility go to \sphinxstylestrong{*Security \textgreater{} Reporting \textgreater{} DoS \textgreater{}
Dashboard (note it may take a few moments for the data to fully
populate).*} You should see a screen similar to the one below.

\sphinxincludegraphics[width=6.50000in,height=3.01597in]{{image53}.png}

Note the real time CPU, RAM, and Throughput stats. When an attack has
stopped the line will stop, so it’s easy to see what attacks are still
active. Feel free to explore the dashboard and the data represented.

In the configuration utility go to \sphinxstylestrong{*Security \textgreater{} Reporting \textgreater{} DoS \textgreater{}
Analysis (note it may take a few moments for the data to fully
populate).*} You should see a screen similar to the one below.

\sphinxincludegraphics[width=6.47361in,height=3.52639in]{{image54}.png}

In the configuration utility go to \sphinxstylestrong{*Security \textgreater{} Reporting \textgreater{} Network \textgreater{}
TCP/IP Errors (note it may take a few moments for the data to fully
populate).*} You should see a screen similar to the one below.

\sphinxincludegraphics[width=6.47361in,height=3.21042in]{{image55}.png}

Examine the other options in the \sphinxstylestrong{*View By*} drop down menu. When you
are finished examining the options go to \sphinxstylestrong{*Security \textgreater{} Overview \textgreater{}
Summary*} screen. Try some of the various options in the top right of
each chart. You can change between \sphinxstylestrong{*Details*}, \sphinxstylestrong{*Line Chart*},
\sphinxstylestrong{*Pie Chart*} and \sphinxstylestrong{*Bar Charts*}. Also note how you can export this
data to CSV or PDF format. Below are some examples of the summaries:

\sphinxincludegraphics[width=6.49514in,height=3.53681in]{{image56}.png}

All of the reports are historical and provide aggregate stats based upon
the selected time period (last day, month, year etc…). In version 11.6
real time DDoS monitoring was added so that an administrator can see
what attacks are currently active, how serious they are, and how long
they have been active. The real time DDoS attack reporting also provides
visibility into the health of the BIG-IP by showing real time CPU, RAM,
and Throughput consumption.

Paste in all of the DDoS attack commands into the Scapy window again. In
the BIG-IP GUI go to \sphinxstylestrong{*Security \textgreater{} Reporting \textgreater{} DoS \textgreater{} Overview
Summary*}.

This concludes the AFM DDoS lab.

\sphinxstylestrong{*Before proceeding, please change the following logging settings for
the remainder of the labs to work correctly:*}

Login to bigip01.agility.com (10.0.0.4).

Navigate to Security \textgreater{} Event Logs \textgreater{} Logging Profiles

Click on global-network

Modify the Network Firewall Publisher to
\sphinxstylestrong{*Log-Publisher-Network-Firewall*}

\sphinxincludegraphics[width=3.61111in,height=2.40972in]{{image57}.png}

Click Update

Navigate to Security \textgreater{} DoS Protection \textgreater{} Device Configuration \textgreater{}
Properties

Modify the Log Publisher to \sphinxstylestrong{*Log-Publisher-Network-DOS-Protection*}

\sphinxincludegraphics[width=3.74306in,height=1.51389in]{{image58}.png}

Click Commit Changes to System


\subsection{Test Access to the Wildcard Virtual Server}
\label{\detokenize{class1/module3/lab4:test-access-to-the-wildcard-virtual-server}}\label{\detokenize{class1/module3/lab4::doc}}\begin{itemize}
\item {} 
Open a new Web browser and access
\sphinxhref{http://10.128.10.223:8081}{*http://10.128.10.223:8081*} \sphinxstylestrong{(this
is expected to fail due to policy)}

\item {} 
Edit the URL to \sphinxhref{https://10.128.10.223}{*https://10.128.10.223*}

\item {} 
Edit the URL to \sphinxhref{http://10.128.10.223}{*http://10.128.10.223*}

\item {} 
Open either Chrome or Firefox and access
\sphinxhref{ftp://10.128.10.223}{*ftp://10.128.10.223*}

\item {} 
Open Putty and access 10.128.10.223

\item {} 
Close all Web browsers and Putty sessions.

\item {} 
Paste in all the DDoS attack commands into the Scapy window again

\end{itemize}


\section{Device Management}
\label{\detokenize{class1/module4/module4:device-management}}\label{\detokenize{class1/module4/module4::doc}}
During this lab, you will configure the BIG-IP system to detect and
report on various network level Denial of Service events. You will then
run simulated attacks against the BIG-IP and verify the mitigation,
reporting and logging of these attacks.


\subsection{BIG-IQ Workflow Overview}
\label{\detokenize{class1/module4/lab1:big-iq-workflow-overview}}\label{\detokenize{class1/module4/lab1::doc}}
\sphinxstylestrong{Statistics Dashboards}

This is the real first step managing data statistics using a DCD (data
collection device) evolving toward a true analytics platform. In this
guide, we will explore setting up and establishing connectivity using
master key to each DCD (data collection device).
\begin{itemize}
\item {} 
Enabling statistics for each functional area as part of the discovery
process. This will allow BIG-IQ to proxy statistics gathered and
organized from each BIG-IP device leveraging F5 Analytics iApp
service
(\sphinxhref{https://devcentral.f5.com/codeshare/f5-analytics-iapp}{*https://devcentral.f5.com/codeshare/f5-analytics-iapp*}).

\item {} 
Configuration and tuning of statistic collections post discovery
allowing the user to focus on data specific to their needs.

\item {} 
Viewing and interaction with statistics dashboard, such as filtering
views, differing time spans, selection and drill down into dashboards
for granular data trends and setting a refresh interval for
collections.

\end{itemize}

\sphinxstylestrong{SSL Certificate Management}

BIG-IQ 5.2 has introduced the ability to manage SSL certificates. From
creating self-signed certificates to creating a CSR (Certificate Signing
Request) provided to a Certificate Authority when applying for a SSL
Certificate. Some features we will cover in this lab:
\begin{itemize}
\item {} 
Importing SSL certificates, key information and PKCS12 “Personal
Information Exchange Syntax Standard” bundles.

\item {} 
When discovering a BIG-IP device, BIG-IQ will import the metadata
from the certificates discovered. These certificates are unmanaged.
BIG-IQ provides the ability to move or convert to a fully managed
certificate by porting SSL certificate source and SSL key properties
into BIG-IQ.

\item {} 
Related to searching to display where SSL certificates are used.

\item {} 
Renew an expired self-signed SSL certificate on BIG-IQ.

\item {} 
Provide a reference for a SSL certificate / key pair to a Server SSL
profile.

\end{itemize}

\sphinxstylestrong{Global Search}

BIG-IQ 5.2 has introduced platform wide search that will allow the user
to globally search for any object or object contents and display
related-to objects. Some features we will cover in this lab:
\begin{itemize}
\item {} 
Search for specific terms across all of BIG-IQ.

\item {} 
Narrow the scope to the search to show “all of an object type”.

\item {} 
Selection of an object to drill into an editable page.

\item {} 
Search for specific CVE-\#\#\#\#-\#\#\#\# in attack signatures documentation
to find an ASM signature.

\end{itemize}

\sphinxstylestrong{Partial Deployment/Partial Restore}

BIG-IQ 5.2 has introduced the flexibility to deploy or restore selective
changes made:
\begin{itemize}
\item {} 
Provides a user the ability to select only changes, out of many, he
or she wants or is approved to deploy during the evaluation process.

\item {} 
As well as the ability to rollback or restore selective changes out
of multiple staged changes.

\end{itemize}

\sphinxstylestrong{New Server SSL Profiles}
\begin{itemize}
\item {} 
Client / Server SSL \textendash{} Enables BIG-IP to initial secure connections to
SSL servers using fully SSL-encapsulated protocol.

\item {} 
HTTP \textendash{} This profile will leverage the header contents to define the
way to manage http traffic through BIG-IP.

\item {} 
Universal / Cookie Persistence -
\begin{itemize}
\item {} 
Universal persistence profile. To persist connections based on the
string.

\item {} 
Cookie-based session persistence. Cookie persistence directs
session requests to the same server based on HTTP cookies that the
BIG-IP system stores in the client’s browser.

\end{itemize}

\end{itemize}

\sphinxstylestrong{Public Facing REST API References and HOWTO Guides}

BIG-IQ 5.2 has introduced documentation that will assist the Engineer
when automating central management tasks or providing integration with
orchestration tools using a REST API using HTTPS. In this lab, we will
explore a couple example API Calls and supporting reference and how-to
documents.
\begin{itemize}
\item {} 
Device Management \textendash{} trust, discover, enable statistics and import
configuration.

\item {} 
Add a policy (firewall) to an application \textendash{} select an existing policy
and reference to a virtual server.

\end{itemize}

\sphinxstylestrong{Licensing Server}

BIG-IQ 5.1 and 5.2 licensing support for four differing pool models.
Using the base registration key and correct SKU, users can enable and
activate BIG-IP virtual editions of types:
\begin{itemize}
\item {} 
Purchased Pools \textendash{} are purchased once, and you assign them to a number
of concurrent BIG-IP VE devices, as defined by the license. These
licenses do not expire. Purchased license pools contain VEP in the
name of the license.

\item {} 
Volume Pools - are prepaid for a fixed number of concurrent devices,
for a set period of time, but have a number of different license
offerings available in the pool. Volume license pools contain VLS in
the name of the license.

\item {} 
Utility License Pools \textendash{} provide the customer the ability to use
licenses as they need them, and true up with F5 for their actual
usage. VE licenses can be granted with usage billing at an hourly,
daily, monthly, or yearly interval. Utility license pools contain
MSP-LOADV in the name of the license.

\item {} 
Registration Key Pools \textendash{} A pool of single standalone BIG-IP virtual
edition registration keys, allowing customers to import their
existing keys and/or import new keys with just the options they
require.

\end{itemize}


\subsection{Dependencies}
\label{\detokenize{class1/module4/lab2:dependencies}}\label{\detokenize{class1/module4/lab2::doc}}\begin{itemize}
\item {} 
The BIG-IP device must be located in your network.

\item {} 
The BIG-IP device must be running a compatible software version.

\item {} 
Enable basic authentication on BIG-IQ using set-basic-auth on in the
shell.

\end{itemize}

\sphinxstylestrong{BIG-IP Versions} AskF5 SOL with this info:

\sphinxurl{https://support.f5.com/kb/en-us/solutions/public/14000/500/sol14592.html}

\begin{sphinxadmonition}{note}{Note:}
Ports 22 and 443 must be open to the BIG-IQ management
address, or any alternative IP address used to add the BIG-IP device to
the BIG-IQ inventory.
\end{sphinxadmonition}


\begin{savenotes}\sphinxattablestart
\centering
\begin{tabulary}{\linewidth}[t]{|T|T|}
\hline

\sphinxstylestrong{Description}
&
\sphinxstylestrong{Minimum BIG-IP version}
\\
\hline
Backup/Restore
&
11.5.0 HF7
\\
\hline
Upgrade - legacy devices
&
10.2.0
\\
\hline
Upgrade - managed devices
&
11.5.0 HF7
\\
\hline
Licensing BIG-IP VE
&
11.5.0 HF7
\\
\hline
Licensing \textendash{} Web-Safe
&
12.0.0
\\
\hline
ADC management
&
11.5.1 HF4
\\
\hline
AFM
&
11.5.2
\\
\hline
Access
&
12.1.0
\\
\hline
ASM
&
11.5.3 HF1
\\
\hline
DNS
&
12.0.0
\\
\hline
\end{tabulary}
\par
\sphinxattableend\end{savenotes}


\subsection{Changes to BIG-IQ User Interface}
\label{\detokenize{class1/module4/lab3:changes-to-big-iq-user-interface}}\label{\detokenize{class1/module4/lab3::doc}}
The user interface in the 5.2 release navigation has changed to a more
UI tab based framework.

In this section, we will go through the main features of the user
interface. Feel free to log into the BIG-IQ device to explore some of
these features in the lab.

After you log into BIG-IQ, you will notice:
\begin{itemize}
\item {} 
A navigation tab model at the top of the screen to display each high
level functional area.

\item {} 
A tree based menu on the left-hand side of the screen to display
low-level functional area for each tab.

\item {} 
A large object browsing and editing area on the right-hand side of
the screen.

\sphinxincludegraphics[width=6.50000in,height=1.44792in]{{image59}.png}

\item {} 
Let us look a little deeper at the different options available in the
bar at the top of the page.

\sphinxincludegraphics[width=6.48958in,height=2.40625in]{{image60}.png}

\item {} 
At the top, each tab describes a high-level functional area for
BIG-IQ central management:

\item {} 
Monitoring \textendash{}Visibility in dashboard format to monitor performance and
isolate fault area.

\item {} 
Configuration \textendash{} Provides configuration editors for each module area.

\item {} 
Deployment \textendash{} Provides operational functions around deployment for
each module area.

\item {} 
Devices \textendash{} Lifecycle management around discovery, licensing and
software install / upgrade.

\item {} 
System \textendash{} Management and monitoring of BIG-IQ functionality.

\item {} 
Overview of left hand navigation for each top-level functional area.

\sphinxincludegraphics[width=6.49028in,height=6.33819in]{{image61}.png}

\end{itemize}

BIG-IQ 5.2 has introduced “\sphinxstylestrong{global search}” which was added to the
BIG-IQ toolbar top here, and will be explored further in this lab.

\sphinxincludegraphics[width=6.50000in,height=0.55208in]{{image62}.png}

Next to the username, there is an icon of a person. If you click on that
icon, a menu appears to allow a user to logout of BIG-IQ.

\sphinxincludegraphics[width=6.50000in,height=0.56250in]{{image63}.png}


\subsection{BIG-IQ Statistics Dashboards}
\label{\detokenize{class1/module4/lab4:big-iq-statistics-dashboards}}\label{\detokenize{class1/module4/lab4::doc}}

\subsubsection{WORKFLOW 1: Setting up of BIG-IQ Data Collection Devices (DCD) and establishing connectivity to BIG-IQ console. (REQUIRED)}
\label{\detokenize{class1/module4/lab4:workflow-1-setting-up-of-big-iq-data-collection-devices-dcd-and-establishing-connectivity-to-big-iq-console-required}}

\paragraph{Objective}
\label{\detokenize{class1/module4/lab4:objective}}
To introduce the user to a DCD, establish connectivity with BIG-IQ
console node to begin data collection task. For the purposes of this lab
the Data Collection Device has already been deployed and licensed with
the appropriate license key (F5-BIQ-LOGNOD101010E-LIC).

Click Add to add a DCD to the BIG-IQ console node.
\begin{itemize}
\item {} 
Log in to the BIG-IQ Console Node (10.0.0.200 admin/401elliottW!)

\item {} 
Under System\(\rightarrow\)BIG-IQ DATA COLLECTION

\item {} 
Select BIG-IQ Data Collection Devices

\item {} 
Click the Add button

\sphinxincludegraphics[width=6.48958in,height=5.20833in]{{image64}.png}

\item {} 
Add the DCD Management IP Address (10.0.0.201), Username: admin,
Password: 401elliottW! and the Data Collection IP Address (self-IP:
10.128.10.201). Data collection port default is 9300. Click the Add
button in the lower right of the screen.

\sphinxincludegraphics[width=5.44792in,height=3.73958in]{{image65}.png}

\item {} 
Adding the DCD will take a minute or two:

\sphinxincludegraphics[width=6.50000in,height=2.03125in]{{image66}.png}

\item {} 
DCD item in UI displayed.
\begin{itemize}
\item {} 
Status \textendash{} State indicator. Green (UP) \textbar{} Yellow (Unhealthy) \textbar{} Red
(Down)

\item {} 
Device name \textendash{} Hostname of DCD (data collection device)

\item {} 
IP Address \textendash{} IP Address of interface used for data collection.

\item {} 
Version \textendash{} Software version of BIG-IQ DCD (data collection device)

\end{itemize}

\end{itemize}

Add device to inventory after DCD has been added to see the user experience around statistics.

We will discover devices 10.0.0.4 using the UI and 10.0.0.5 using REST
and enable statistic collection for these BIG-IP’s.

Click Add to add a device to the BIG-IQ console.
\begin{itemize}
\item {} 
Log in to the BIG-IQ Console Node (10.0.0.200 admin/401elliottW!)

\item {} 
Under Device\(\rightarrow\)Add Device
\begin{quote}

\sphinxincludegraphics[width=5.85291in,height=2.35171in]{{image67}.png}

-Complete the form for the device add using IP 10.0.0.4, username:
admin, password: 401elliottW!

\sphinxincludegraphics[width=5.90178in,height=2.97140in]{{image68}.png}
\end{quote}

\end{itemize}

\sphinxincludegraphics[width=5.51646in,height=4.65893in]{{image69}.png}

To discover 10.0.0.5, use the POSTMan collection labeled “Service Provider Specialist Event - Lab 4”. Please note you may have to manually import the ADC service due to a conflict. Conflict resolution is capable via the API however; outside of the scope of this lab. For additional details please reference the API documentation located here:

\sphinxurl{http://bigiq-cm-restapi-reference.readthedocs.io/en/latest/HowToGuides/Trust/Trust.html}
\begin{itemize}
\item {} 
Complete the Import (current-configuration copy to
working-configuration on BIG-IQ) for LTM and AFM for both BIG-IP’s.
For any conflict resolution use BIG-IP as the source of truth

\sphinxincludegraphics[width=2.57055in,height=1.69135in]{{image70}.png}

\end{itemize}

Navigate to the monitoring dashboards to validate that statistics are being collected and displayed for the BIG-IP devices.
\begin{itemize}
\item {} 
Navigate to Monitoring\(\rightarrow\)Dashboards\(\rightarrow\) Device\(\rightarrow\) Health to verify that the
graphs are populated.

\sphinxincludegraphics[width=5.84302in,height=4.64525in]{{image71}.png}

\item {} 
If you don’t see data, raise your hand to get some help.

\item {} 
We are going to move on to other parts of the lab while we collect
some stats and then we will circle back when we have more data to
work with.

\end{itemize}


\subsection{WORKFLOW 2: Creating a Backup Schedule}
\label{\detokenize{class1/module4/lab4:workflow-2-creating-a-backup-schedule}}
BIG-IQ is capable of centrally backing up and restoring all of the
BIG-IP devices it manages. To create a simple backup schedule, follow
the following steps.
\begin{enumerate}
\item {} 
Click on the \sphinxstylestrong{Back Up \& Restore} submenu in the Devices header.
\begin{quote}

\sphinxincludegraphics[width=6.50000in,height=2.80208in]{{image72}.png}
\end{quote}

\item {} 
Expand the \sphinxstylestrong{Back Up and Restore} menu item found on the left and
click on \sphinxstylestrong{Backup Schedules}\sphinxincludegraphics[width=2.28056in,height=1.23889in]{{image73}.png}

\item {} 
Click the \sphinxstylestrong{Create} button
\begin{quote}

\sphinxincludegraphics[width=2.00000in,height=1.47917in]{{image74}.png}
\end{quote}

\item {} 
Fill out the Backup Schedule using the following settings:
\begin{itemize}
\item {} 
\sphinxstylestrong{Name:} Nightly

\item {} 
\sphinxstylestrong{Local Retention Policy:} Delete local backup copy 1 day after creation

\item {} 
\sphinxstylestrong{Backup Frequency:} Daily

\item {} 
\sphinxstylestrong{Start Time:} 00:00 Eastern Daylight Time

\item {} 
\sphinxstylestrong{Devices: Groups:} All BIG-IP Devices

\item {} 
Your screen should look similar to the one below.

\sphinxincludegraphics[width=6.50000in,height=4.85417in]{{image75}.png}

\end{itemize}

\item {} 
Click \sphinxstylestrong{Save} to save the scheduled backup job.

\item {} 
Optionally feel free to select the newly created schedule and select
“Back Up Now” to immediately backup the devices.
\begin{enumerate}
\item {} 
When completed the backups will be listed under the Backup Files
section

\end{enumerate}

\end{enumerate}


\subsection{WORKFLOW 3: Uploading QKViews to iHealth for a support case}
\label{\detokenize{class1/module4/lab4:workflow-3-uploading-qkviews-to-ihealth-for-a-support-case}}
BIG-IQ can now push QKViews from managed devices to ihealth.f5.com and
provide a link to the report of heuristic hits based on the QKView.
These QKView uploads can be performed ad-hoc or as part of a F5 support
case. If a support case is specified in the upload job, the QKView(s)
will automatically be associated/linked to the support case. In addition
to the link to the report, the QKView data is accessible at
ihealth.f5.com to take advantage of other iHealth features like the
upgrade advisor.
\begin{enumerate}
\item {} 
Navigate to \sphinxstylestrong{Monitoring} \sphinxstylestrong{\(\rightarrow\) Reports} \(\rightarrow\) \sphinxstylestrong{Device} \(\rightarrow\) \sphinxstylestrong{iHealth \(\rightarrow\)
Configuration}
\begin{quote}

\sphinxincludegraphics[width=6.50000in,height=2.70000in]{{image76}.png}
\end{quote}

\item {} 
Add Credentials to be used for the QKView upload and report
retrieval. Click the Add button under Credentials.

\sphinxincludegraphics[width=1.88472in,height=0.92639in]{{image77}.png}

\item {} 
Fill in the credentials that you used to access \sphinxurl{https://ihealth.f5.com}:
\begin{itemize}
\item {} 
Name: Give the credentials a name to be referenced in BIG-IQ

\item {} 
Username: \textless{}Username you use to access iHealth.f5.com\textgreater{}

\item {} 
Password: \textless{}Password you use to access iHealth.f5.com\textgreater{}

\end{itemize}

\sphinxincludegraphics[width=4.50000in,height=2.85417in]{{image78}.png}

\item {} 
Click the Test button to validate that your credentials work.

\item {} 
Click the Save \& Close button in the lower right.

\item {} 
Click the Uploads button in the BIG-IP iHealth menu.

\item {} 
Click the Upload button to select which devices we need to upload
QKViews from Case123456

\item {} 
Fill in the fields to upload the QKViews to iHealth.
\begin{itemize}
\item {} 
Name: CaseC123456

\item {} 
F5 Support Case Number: C123456

\item {} 
Credentials: \textless{}Select the credentials you just stored in step 5\textgreater{}

\item {} 
Devices: Move all devices from Available to Selected

\end{itemize}

\sphinxincludegraphics[width=6.50000in,height=3.10000in]{{image79}.png}

\item {} 
Click the Upload button in the lower right.

\item {} 
Click on the name of your upload job to get more details

\sphinxincludegraphics[width=2.82222in,height=0.74931in]{{image80}.png}

\item {} 
Observe the progress of the QKView creation, retrieval, upload,
processing, and reporting. This operation can take some time, so you
may want to move on to the next exercise and come back.

\item {} 
Once a job reaches the Finished status, click on the Reports menu to
review the report.

\item {} 
Click on the Download PDF link to view each of the reports.

\sphinxincludegraphics[width=6.50000in,height=2.89583in]{{image81}.png}

\item {} 
Open a browser window/tab to \sphinxurl{https://ihealth.f5.com}

\item {} 
Log in with the same credentials that you saved in step 5.

\item {} 
Observe the full QKViews that are available in iHealth for further
use with items like the Upgrade Advisor.

\sphinxincludegraphics[width=6.47361in,height=0.84236in]{{image82}.png}

\end{enumerate}


\subsection{BIG-IQ Global Search and Related to Search}
\label{\detokenize{class1/module4/lab5::doc}}\label{\detokenize{class1/module4/lab5:big-iq-global-search-and-related-to-search}}

\subsubsection{WORKFLOW 1: BIG-IQ Global Search and Related to Search (REQUIRED)}
\label{\detokenize{class1/module4/lab5:workflow-1-big-iq-global-search-and-related-to-search-required}}

\paragraph{Objective}
\label{\detokenize{class1/module4/lab5:objective}}
To introduce the user to BIG-IQ global search which will provide a
product wide index to all configuration objects and supporting
properties.


\paragraph{Global Search}
\label{\detokenize{class1/module4/lab5:global-search}}\begin{itemize}
\item {} 
Select the Global Search Icon in the upper right corner of the top
panel.

\sphinxincludegraphics[width=4.06250in,height=1.11458in]{{image83}.png}

\item {} 
Enter search terminology.

\sphinxincludegraphics[width=6.50000in,height=0.64583in]{{image84}.png}

\item {} 
Produces results.

\item {} 
Select to the object to preview by drilling into editor page.

\sphinxincludegraphics[width=6.48958in,height=3.41667in]{{image85}.png}

\item {} 
Narrowing the search results

\item {} 
Click the down arrow next to the search term and observe that you
have various options to scope your search

\item {} 
Update your search to only return pool members that match the
search term and refresh your results

\sphinxincludegraphics[width=6.50000in,height=3.05208in]{{image86}.png}

\end{itemize}

\sphinxincludegraphics[width=6.72676in,height=1.88954in]{{image87}.png}

\begin{sphinxadmonition}{note}{Note:}
This can be especially useful if you are running a traceroute and want
to narrow down the results to the IP and/or hostname of the firewall in
the traceroute. Try searching for the IP 10.128.10.11 with a filter of
Object Type and SelfIP (hit you can start typing what you’re looking
for). Your results should look similar the following:
\end{sphinxadmonition}

\sphinxincludegraphics[width=6.50000in,height=1.00000in]{{image88}.png}


\subsection{BIG-IQ Statistics Dashboards (Continued)}
\label{\detokenize{class1/module4/lab6:big-iq-statistics-dashboards-continued}}\label{\detokenize{class1/module4/lab6::doc}}
\begin{sphinxadmonition}{note}{Note:}
Now that some time has passed, we should have more statistics to review and
interact with.
\end{sphinxadmonition}


\subsubsection{WORKFLOW 1: Reviewing the data in the dashboards}
\label{\detokenize{class1/module4/lab6:workflow-1-reviewing-the-data-in-the-dashboards}}
Navigate to Monitoring\(\rightarrow\)Dashboards\(\rightarrow\)Local Traffic
\begin{itemize}
\item {} 
Click through all the stats dashboards and see the metrics that are
gathered

\sphinxincludegraphics[width=2.19764in,height=2.38512in]{{image89}.png}

\end{itemize}


\subsubsection{WORKFLOW 2: Interacting with the data in the dashboards}
\label{\detokenize{class1/module4/lab6:workflow-2-interacting-with-the-data-in-the-dashboards}}\begin{itemize}
\item {} 
You can narrow the scope of what is graphed by selecting an object
or objects from the selection panels on the right. For example, if
you only want to see data from BIG-IP01, you can click on it to
filter the data.

\sphinxincludegraphics[width=2.40653in,height=3.14682in]{{image90}.png}

\item {} 
You can create complex filters by making additional selections in
other panels

\item {} 
You can zoom in on a time, by selecting a section of a graph or
moving the slider at the top of the page

\sphinxincludegraphics[width=4.91605in,height=1.91643in]{{image91}.png}

or

\sphinxincludegraphics[width=4.48902in,height=0.91655in]{{image92}.png}

\item {} 
All the graphs update to the selected time.

\item {} 
You can change how far in the data you want to look back by using
the selection in the upper left (note you may need to let some time
elapse before this option becomes available)

\sphinxincludegraphics[width=2.40595in,height=2.81215in]{{image93}.png}

\item {} 
Creating a comparison chart

\item {} 
You can select multiple objects of the same type and create a
comparison chart. Select the objects in the right-hand selector,
right click, and select Add Comparison Chart

\sphinxincludegraphics[width=2.54135in,height=2.19764in]{{image94}.png}

\item {} 
\begin{DUlineblock}{0em}
\item[] The chart will appear at the top of the page. You can choose what
metric you want to see in the chart
\item[] \sphinxincludegraphics[width=6.39503in,height=3.32250in]{{image95}.png}
\end{DUlineblock}

\item {} 
Viewing the data in tabular form

\item {} 
You can open the selector panel on the right to view and interact
with the data in tabular form. Double click the tab at the top of
the selector panel to open up the tabular view.

\sphinxincludegraphics[width=2.35387in,height=1.03112in]{{image96}.png}

\sphinxincludegraphics[width=6.50000in,height=1.65486in]{{image97}.png}

\item {} 
Now you can view the averages, sort the columns, and add and remove
columns from the view, by right clicking on the table

\sphinxincludegraphics[width=4.09324in,height=2.54135in]{{image98}.png}

\end{itemize}


\subsubsection{WORKFLOW 3: Getting to the statistics from the configuration/property pages for an object}
\label{\detokenize{class1/module4/lab6:workflow-3-getting-to-the-statistics-from-the-configuration-property-pages-for-an-object}}\begin{itemize}
\item {} 
If you know that you only want to see stats for an object, you can
launch the stats page from the configuration table or properties
page.

\item {} 
Navigate to Configuration\(\rightarrow\)Local Traffic \(\rightarrow\) Virtual Servers

\item {} 
Select the object you want to see stats for, click the more button,
and click view statistics

\sphinxincludegraphics[width=4.78065in,height=2.06224in]{{image99}.png}

Or open the properties page of a virtual server and click the view
statistics button in the upper right

\sphinxincludegraphics[width=1.37483in,height=0.87489in]{{image100}.png}

\item {} 
The launched stats page will be filtered to the object or objects
you selected.

\sphinxincludegraphics[width=2.65592in,height=2.01017in]{{image101}.png}

\end{itemize}


\subsection{BIG-IQ REST API Documentation}
\label{\detokenize{class1/module4/lab7:big-iq-rest-api-documentation}}\label{\detokenize{class1/module4/lab7::doc}}

\subsubsection{References (DevCentral How-to and supporting references):}
\label{\detokenize{class1/module4/lab7:references-devcentral-how-to-and-supporting-references}}
\sphinxurl{https://devcentral.f5.com/wiki/BIGIQ.HomePage.ashx}


\section{Network Security (AFM) Management Workflows}
\label{\detokenize{class1/module5/module5:network-security-afm-management-workflows}}\label{\detokenize{class1/module5/module5::doc}}

\subsection{Managing AFM from BIG-IQ}
\label{\detokenize{class1/module5/lab1:managing-afm-from-big-iq}}\label{\detokenize{class1/module5/lab1::doc}}

\subsubsection{WORKFLOW 1: Managing AFM from BIG-IQ}
\label{\detokenize{class1/module5/lab1:workflow-1-managing-afm-from-big-iq}}
In this lab, you will create all the components of a firewall policy.
Port lists and address lists are the building blocks of firewall
policies. They can be nested inside of each other to make address and
port management easier, or policies can be created without using the
lists at all. In this example, you’ll use the lists to see how they
work. Once created, you will deploy the configuration to two BIG-IP
units.

\sphinxstylestrong{Objective}
\begin{itemize}
\item {} 
Create a simple firewall policy and dependent objects (address and
port list)

\item {} 
Deploy new firewall configuration to BIG-IP

\end{itemize}

\sphinxstylestrong{Lab Requirements}
\begin{itemize}
\item {} 
Web UI access to BIG-IQ

\end{itemize}


\paragraph{Task 1 \textendash{} Create Port List}
\label{\detokenize{class1/module5/lab1:task-1-create-port-list}}
On \sphinxstylestrong{BIGIQ1}: (\sphinxhref{about:blank}{*https://10.0.0.200)*}

Navigate to the \sphinxstylestrong{Configuration} tab from the top menu of \sphinxstylestrong{BIG-IQ}
then to \sphinxstylestrong{Security \(\rightarrow\) Network Security}.

\sphinxincludegraphics[width=3.38542in,height=4.42708in]{{image102}.png}

Click \sphinxstylestrong{Port Lists} from the left side navigation menu.

Click the \sphinxstylestrong{Create} button.

On the \sphinxstylestrong{Propertie}s tab, type in \sphinxstylestrong{HTTP\_HTTPS} for \sphinxstylestrong{Name.}

Click the \sphinxstylestrong{Ports} tab.

Create the port list with the below information.


\begin{savenotes}\sphinxattablestart
\centering
\begin{tabulary}{\linewidth}[t]{|T|T|T|}
\hline

Type
&
Ports
&
Description
\\
\hline
Port
&
80
&
HTTP
\\
\hline
Port
&
443
&
HTTPS
\\
\hline
\end{tabulary}
\par
\sphinxattableend\end{savenotes}

Click on the \sphinxstylestrong{+} to add additional ports

\sphinxincludegraphics[width=6.50000in,height=1.10000in]{{image103}.png}

Click \sphinxstylestrong{Save \& Close} when finished.

To create a port list via the API, follow the POSTMan Collection
“Service Provider Specialist Event - Lab 5”, using Step 1 \textendash{} Create New
Port List. It is important to note the value returned within the
self-link as shown below:

\sphinxincludegraphics[width=6.50000in,height=2.76042in]{{image104}.png}

This value will be assigned to the environment variable AFM\_Port\_ID.

To modify the environment variables, click on the “eye” icon located
in the top right section of POSTMan and select edit. The screen shot
following shows an example of this screen:

\sphinxincludegraphics[width=4.56250in,height=5.16667in]{{image105}.png}


\paragraph{Task 2 \textendash{} Create Address List}
\label{\detokenize{class1/module5/lab1:task-2-create-address-list}}
Click on the \sphinxstylestrong{Address Lists} from the left navigation menu

Click the \sphinxstylestrong{Create} button

In Properties, type in \sphinxstylestrong{Trusted\_Clients} for Name

The ability to Pin address to a device is also new in 5.2. This feature
allows objects to remain on a device even if they are orphaned and/or
not currently in use in a policy on the “pinned” device.

Click the \sphinxstylestrong{Addresses} tab

Create a new address list with the below information


\begin{savenotes}\sphinxattablestart
\centering
\begin{tabulary}{\linewidth}[t]{|T|T|T|}
\hline

\sphinxstylestrong{Type}
&
\sphinxstylestrong{Addresses}
&
\sphinxstylestrong{Description}
\\
\hline
\sphinxstylestrong{Address}
&
10.128.10.0/24
&
Internal Network
\\
\hline
\sphinxstylestrong{Address}
&
172.16.16.99
&
Internal Client 2
\\
\hline
\end{tabulary}
\par
\sphinxattableend\end{savenotes}

Click on the \sphinxstylestrong{+} to add additional addresses

\sphinxincludegraphics[width=6.50000in,height=1.10000in]{{image106}.png}

Click \sphinxstylestrong{Save and Close} when finished.

To create an address list via the API, follow the POSTMan Collection
“Service Provider Specialist Event - Lab 5”, using Step 2 \textendash{} Create New
Address List. It is important to note the value returned within the
self-link as shown below:

\sphinxincludegraphics[width=6.50000in,height=2.66667in]{{image107}.png}

This value will be assigned to the environment variable
AFM\_Address\_ID.


\paragraph{Task 3 \textendash{} Create Rule List}
\label{\detokenize{class1/module5/lab1:task-3-create-rule-list}}
Click on the \sphinxstylestrong{Rule Lists} from the left navigation menu.

Click the \sphinxstylestrong{Create} button.

On the Properties tab, type in \sphinxstylestrong{Rule\_List\_Allow\_Trusted} for Name.

Click the \sphinxstylestrong{Rules} tab.

Click \sphinxstylestrong{Create Rule} button.

Click on the pencil (edit rule) of the newly created rule listed with
\sphinxstylestrong{Id} of \sphinxstylestrong{1.}

Create a new rule with the below information.


\begin{savenotes}\sphinxattablestart
\centering
\begin{tabulary}{\linewidth}[t]{|T|T|T|}
\hline

\sphinxstylestrong{Name}
&&
Rule\_Allow\_Trusted
\\
\hline
{\color{red}\bfseries{}**}Source Address **
&
\sphinxstylestrong{Address List}
&
Trusted\_Clients
\\
\hline
\sphinxstylestrong{Source Port}
&
{\color{red}\bfseries{}**}Port **
&
Any
\\
\hline
\sphinxstylestrong{Source VLAN}
&&
Any
\\
\hline
\sphinxstylestrong{Destination Address}
&
\sphinxstylestrong{Address}
&
Any
\\
\hline
\sphinxstylestrong{Destination Port}
&
\sphinxstylestrong{Port List}
&
HTTP\_HTTPS
\\
\hline
\sphinxstylestrong{Action}
&
\sphinxstylestrong{Accept}
&
Accept
\\
\hline
\sphinxstylestrong{Protocol}
&
\sphinxstylestrong{TCP}
&
TCP
\\
\hline
\sphinxstylestrong{State}
&&
enabled
\\
\hline
\sphinxstylestrong{Log}
&&
checked
\\
\hline
\end{tabulary}
\par
\sphinxattableend\end{savenotes}

\sphinxincludegraphics[width=6.48958in,height=0.67708in]{{image108}.png}

Click \sphinxstylestrong{Save \& Close} when finished.

To create a rule list via the API, follow the POSTMan Collection
“Service Provider Specialist Event - Lab 5”, using Step 3 \textendash{} Create New
Rule. It is important to note the value returned within the self-link as
shown below:

\sphinxincludegraphics[width=6.50000in,height=2.00000in]{{image109}.png}

This value will be assigned to the environment variable AFM\_Rule\_ID.

To create a rule within the rule list via the API, follow the
POSTMan Collection “Service Provider Specialist Event - Lab 5”,
using Step 4 \textendash{} Create New Rule List.


\paragraph{Task 4 \textendash{} Create Firewall Policy}
\label{\detokenize{class1/module5/lab1:task-4-create-firewall-policy}}
Click on \sphinxstylestrong{Firewall Policies} from the left navigation menu.

Click the \sphinxstylestrong{Create} button.

On the Properties tab, type in \sphinxstylestrong{Policy\_Forward} for Name.

On Pin Policy to Device(s), move bigip1.agility.f5.com to Selected.

Click the Rules tab.

Click the \sphinxstylestrong{Add Rule List} button.

Select the checkbox for \sphinxstylestrong{Rule\_Allowed\_Trusted.}

Click \sphinxstylestrong{Add} button.

You will see the new policy listed as shown below.

\sphinxincludegraphics[width=6.48958in,height=3.98958in]{{image110}.png}

Click on drop down arrow to verify our rule within the rule list is
there.

\sphinxincludegraphics[width=6.50000in,height=0.87500in]{{image111}.png}

Click \sphinxstylestrong{Create Rule} button

Click on the pencil (edit rule) of the newly created rule listed with
\sphinxstylestrong{Id} of \sphinxstylestrong{2.}

Create a new rule with the below information.


\begin{savenotes}\sphinxattablestart
\centering
\begin{tabulary}{\linewidth}[t]{|T|T|T|}
\hline

\sphinxstylestrong{Name}
&&
Rule\_Drop\_Everything\_Else
\\
\hline
{\color{red}\bfseries{}**}Source Address **
&
\sphinxstylestrong{Address}
&
Any
\\
\hline
\sphinxstylestrong{Source Port}
&
{\color{red}\bfseries{}**}Port **
&
Any
\\
\hline
\sphinxstylestrong{Source VLAN}
&&
Any
\\
\hline
\sphinxstylestrong{Destination Address}
&
\sphinxstylestrong{Address List}
&
Any
\\
\hline
\sphinxstylestrong{Destination Port}
&
\sphinxstylestrong{Port List}
&
Any
\\
\hline
\sphinxstylestrong{Action}
&&
\sphinxstyleemphasis{drop}
\\
\hline
\sphinxstylestrong{Protocol}
&&
\sphinxstyleemphasis{any}
\\
\hline
\sphinxstylestrong{State}
&&
\sphinxstyleemphasis{enabled}
\\
\hline
\sphinxstylestrong{Log}
&&
\sphinxstyleemphasis{checked}
\\
\hline
\end{tabulary}
\par
\sphinxattableend\end{savenotes}

Click the \sphinxstylestrong{Save and Close} button at the top.

You should see the policy with the new rule as shown below.

\sphinxincludegraphics[width=6.47917in,height=0.84375in]{{image112}.png}

To create a policy via the API, follow the POSTMan Collection “Service
Provider Specialist Event - Lab 5”, using Step 5 \textendash{} Create New Policy. It
is important to note the value returned within the self-link as shown
below:

\sphinxincludegraphics[width=6.50000in,height=2.09375in]{{image113}.png}

This value will be assigned to the environment variable AFM\_Policy\_ID.

To reference a rule within the policy via the API, follow the
POSTMan Collection “Service Provider Specialist Event - Lab 5”,
using Step 6 \textendash{} Create New Rule Reference.

To create a drop rule within the policy via the API, follow the
POSTMan Collection “Service Provider Specialist Event - Lab 5”,
using Step 7 \textendash{} Create Drop Rule in Policy.
\sphinxstyleemphasis{Task 5 \textendash{} Assign the Firewall Policy to a Context.}

In this task, you will take the policy you created above and apply
it to a route domain on a BIG-IP. Typically, the route domain you
apply firewall policies to has a wildcard virtual server that you
forward all traffic through (as opposed to a standard single port
virtual server that only allows specific traffic). This type of
configuration is like the more classic firewall deployment.

In the left navigation menu, click \sphinxstylestrong{contexts}, then chose 0 for device
bigip1.agility.f5.com

\sphinxincludegraphics[width=6.48958in,height=3.20833in]{{image114}.png}

From the \sphinxstylestrong{Shared Objects} panel at the bottom of the screen, \sphinxstyleemphasis{grab}
the \sphinxstylestrong{Policy\_Forward} and \sphinxstyleemphasis{drag} it to the \sphinxstylestrong{Enforced Firewall
Policy} shaded area. The policy should then appear in the \sphinxstylestrong{Enforced
Firewall Policy} section. Alternatively, delete the existing policy
(Common/rd\_0\_policy) by clicking the x, then select \sphinxstylestrong{Add Enforce
Firewall Policy} and select \sphinxstylestrong{Policy\_Forward} and click \sphinxstylestrong{Add.}

\sphinxincludegraphics[width=6.48958in,height=3.67708in]{{image115}.png}

Click the \sphinxstylestrong{Save \& Close} button.

At this point, the policy is assigned to the route domain in the BIG-IQ
configuration, but the configuration has \sphinxstylestrong{not} been deployed/pushed to
the BIG-IP units yet.

To assign a policy via the API, follow the POSTMan Collection “Service
Provider Specialist Event - Lab 5”, using Step 8: Get bigip02 Contexts.
This call will list all the firewall contexts using a filter for just
route-domains. You will need to copy the “id” assigned to bigip02
route-domain as exampled by the following:

\sphinxincludegraphics[width=6.81719in,height=1.78363in]{{image116}.png}

This value will be assigned to the environment variable bigip02-rd0id.

To assign the policy to RD0 via the API, follow the POSTMan
Collection “Service Provider Specialist Event - Lab 5”, using Step
9: Apply Policy to RD0.


\paragraph{Task 6 \textendash{} Deploy the Firewall Policy and related configuration objects}
\label{\detokenize{class1/module5/lab1:task-6-deploy-the-firewall-policy-and-related-configuration-objects}}
Now that the desired firewall configuration has been created on the
BIG-IQ, you need to deploy it to the BIG-IP. In this task, you create
the deployment, verify it, and deploy it.

From the top navigation bar, click on \sphinxstylestrong{Deployments}.

Click on the \sphinxstylestrong{EVALUATE \& DEPLOY} section on the left to expand it.

Click on \sphinxstylestrong{Network Security} in the expansion.

\sphinxincludegraphics[width=6.50000in,height=4.09375in]{{image117}.png}

Click on the top Create button under Evaluations

Give your evaluation a name (ex: \sphinxstylestrong{deploy\_afm1}).

Evaluation \sphinxstylestrong{Source} should be \sphinxstylestrong{Current Changes} (default).

Source Scope should be \sphinxstylestrong{All Changes} (default)

Target Device(s) should be \sphinxstylestrong{Device}.

Select bigip1.agility.f5.com from the list of Available devices and move
it to Selected.

\sphinxincludegraphics[width=6.50000in,height=4.87500in]{{image118}.png}

Click the \sphinxstylestrong{Create} button at the bottom right of the page.

You should be redirected to the main \sphinxstylestrong{Evaluate and Deploy} page.
\begin{itemize}
\item {} 
This will start the evaluation process in which BIG-IQ compares its
working configuration to the configuration active on each BIG-IP.
This can take a few moments to complete.

\end{itemize}

The \sphinxstylestrong{Status} section should be dynamically updating… (What states do
you see?)

Once the status shows \sphinxstylestrong{Evaluation Complete} you can view the
evaluation results.
\begin{itemize}
\item {} 
Before selecting to deploy, feel free to select the differences
indicated to see the proposed deployment changes. This is your check
before making changes on a BIG-IP.

\end{itemize}

Click the number listed under \sphinxstylestrong{Differences \textendash{} Firewall}.

Scroll through the list of changes to be deployed.

Click on a few to review in more detail.

\sphinxincludegraphics[width=6.70833in,height=3.37500in]{{image119}.png}

What differences do you see from the \sphinxstylestrong{Deployed on BIG-IP} section
and on \sphinxstylestrong{BIG-IQ}?

Click \sphinxstylestrong{Cancel}.

Deploy your changes by checking the box next to your evaluation
\sphinxstylestrong{deploy\_afm1}.

With the box checked, click the \sphinxstylestrong{Deploy} button.

Your evaluation should move to the \sphinxstylestrong{Deployments} section.

After deploying, the status should change to \sphinxstylestrong{Deployment Complete}.
\begin{itemize}
\item {} 
This will take a moment to complete. Once completed, log in to the
BIG-IP and verify that the changes have been deployed to the AFM
configuration.

\end{itemize}

To deploy the changes via the API, follow the POSTMan Collection
“Service Provider Specialist Event - Lab 5”, using Step 10: Deploy
Policy to bigip02. This call will deploy only the changes made to
bigip02

Congratulations, you just deployed your first AFM policy via BIG-IQ!

Review the configuration deployed to the BIG-IP units.

On \sphinxstylestrong{BIGIP1}: (\sphinxhref{about:blank}{*https://10.0.04)*}

Navigate to Security \textgreater{} Network Firewall \textgreater{} Policies.

Click on Policy\_Forward.

Are the two rules you created in BIG-IQ listed for this newly deployed
firewall policy?

\sphinxincludegraphics[width=5.31250in,height=1.65625in]{{image120}.png}

Navigate to Network \textgreater{} Route Domains

Click on route domain 0.

Click on the \sphinxstylestrong{Security} tab, click on \sphinxstylestrong{Policies} in the drop down.

What policy is deployed to this route domain?

Are the correct firewall rules applied to this route domain from the
policy you associated to it?

\sphinxincludegraphics[width=4.34028in,height=2.38194in]{{image121}.png}


\subsubsection{Test Access to the Wildcard Virtual Server}
\label{\detokenize{class1/module5/lab1:test-access-to-the-wildcard-virtual-server}}\begin{itemize}
\item {} 
Open a new Web browser and access
\sphinxhref{http://10.128.10.223:8081}{*http://10.128.10.223:8081*} (this is
expected to fail as are some others)

\item {} 
Edit the URL to \sphinxhref{https://10.128.10.223}{*https://10.128.10.223*}

\item {} 
Edit the URL to \sphinxhref{http://10.128.10.223}{*http://10.128.10.223*}

\item {} 
Open either Chrome or Firefox and access
\sphinxhref{ftp://10.128.10.223}{*ftp://10.128.10.223*}

\item {} 
Open Putty and access 10.128.10.223

\item {} 
Close all Web browsers and Putty sessions.

\end{itemize}


\subsection{Service Policies and Timer Policies from BIG-IQ}
\label{\detokenize{class1/module5/lab2::doc}}\label{\detokenize{class1/module5/lab2:service-policies-and-timer-policies-from-big-iq}}

\subsubsection{WORKFLOW 2: Service Policies and Timer Policies from BIG-IQ}
\label{\detokenize{class1/module5/lab2:workflow-2-service-policies-and-timer-policies-from-big-iq}}
In this lab, you’ll be creating a timer policy in a service policy and
associating the service policy to a firewall rule. This allows you to
control the idle connection time before a connection is removed from the
state table. This control within AFM is a new feature in BIG-IP version
12.0+

\sphinxstylestrong{Objective:}
\begin{itemize}
\item {} 
Create a Timer Policy

\item {} 
Create a Service Policy

\item {} 
Associate Service Policy to a firewall rule

\item {} 
Deploy to both BIG-IP units

\end{itemize}

\sphinxstylestrong{Lab Requirements:}
\begin{itemize}
\item {} 
Web UI access to BIG-IQ

\end{itemize}


\paragraph{Task 1 \textendash{} Create a Timer Policy}
\label{\detokenize{class1/module5/lab2:task-1-create-a-timer-policy}}
On \sphinxstylestrong{BIGIQ1}: (\sphinxhref{about:blank}{*https://10.0.0.200)*}

Navigate to Configuration\(\rightarrow\) Security \(\rightarrow\) Network Security \(\rightarrow\) Timer Policies.

\sphinxincludegraphics[width=3.53125in,height=6.69792in]{{image122}.png}

In the left navigation, click on \sphinxstylestrong{Timer Policies}.

Click \sphinxstylestrong{Create}.

In \sphinxstylestrong{Name} field, type \sphinxstyleemphasis{timer\_tcp\_60\_min}.

Click \sphinxstylestrong{Rules} tab.

Click \sphinxstylestrong{Create Rule} button.

Click the pencil next to the new rule to modify it.

Create the new rule with the below information.


\begin{savenotes}\sphinxattablestart
\centering
\begin{tabulary}{\linewidth}[t]{|T|T|T|}
\hline

\sphinxstylestrong{Name}
&
\sphinxstylestrong{timer\_rule\_tcp\_60\_min}
&\\
\hline
{\color{red}\bfseries{}**}Protocol **
&
\sphinxstylestrong{tcp}
&\\
\hline
\sphinxstylestrong{Destination Ports}
&
\sphinxstylestrong{Port Range}
&
\sphinxstylestrong{1 \textendash{} 65535}
\\
\hline
\sphinxstylestrong{Idle Timeouts}
&
\sphinxstylestrong{Specify…}
&
\sphinxstylestrong{3600}
\\
\hline
\sphinxstylestrong{Description}
&
\sphinxstylestrong{Allow TCP connections to idle for 60 minutes}
&\\
\hline
\end{tabulary}
\par
\sphinxattableend\end{savenotes}

Click the \sphinxstylestrong{Save and Close} button at the bottom.


\paragraph{Task 2 \textendash{} Create a Service Policy}
\label{\detokenize{class1/module5/lab2:task-2-create-a-service-policy}}
In the left navigation, click on \sphinxstylestrong{Service Policies}.

Click \sphinxstylestrong{Create} button.

In the \sphinxstylestrong{Name} field, fill in \sphinxstylestrong{policy\_timer}.

On the Timer Policy drop down, select \sphinxstylestrong{/commom/timer\_tcp\_60\_min}.

On Pin Policy to Device(s), move \sphinxstylestrong{bigip1.agility.f5.com}.

Click the \sphinxstylestrong{Save and Close} button at the bottom right.


\paragraph{Task 3 \textendash{} Associate Service Policy to Firewall Rule}
\label{\detokenize{class1/module5/lab2:task-3-associate-service-policy-to-firewall-rule}}
In the left navigation, click on \sphinxstylestrong{Rule Lists}.

Select Rule List named \sphinxstylestrong{Rule\_List\_Allow\_Trusted}.

Click on rule 1 named \sphinxstylestrong{Rule\_Allow\_Trusted} to enter rule
modification mode.

Scroll to the far right.

Under Service Policy field, type in \sphinxstylestrong{policy\_timer}.

Click \sphinxstylestrong{Save} button at the bottom.

Validate \sphinxstylestrong{policy\_timer} is listed under \sphinxstylestrong{Service Policy} on the
rule.

\sphinxincludegraphics[width=2.30000in,height=3.20000in]{{image123}.png}

Click \sphinxstylestrong{Save \& close} button at the top.


\paragraph{Task 4 \textendash{} Deploy the Service Policy and related configuration objects}
\label{\detokenize{class1/module5/lab2:task-4-deploy-the-service-policy-and-related-configuration-objects}}
Now that the desired timer and service policy configuration has been
created on the BIG-IQ, you need to deploy it to the BIG-IP units. In
this task, you create the deployment, verify it, and deploy it.

From the top navigation bar, click on \sphinxstylestrong{Deployment}.

Click on the \sphinxstylestrong{EVALUATE \& DEPLOY} section on the left to expand it.

Click on \sphinxstylestrong{Network Security} in the expansion.

Click on the top Create button under Evaluation

Give your evaluation a name (ex: \sphinxstylestrong{deploy\_afm2}).

Evaluation \sphinxstylestrong{Source} should be \sphinxstylestrong{Current Changes} (default).

Source Scope should be \sphinxstylestrong{All Changes}

Evaluation \sphinxstylestrong{Target} should be \sphinxstylestrong{Device}.

Select bigip1.agility.f5.com from the list of Available devices and move
it to Selected.

\sphinxincludegraphics[width=5.60000in,height=5.50000in]{{image124}.png}

Click the \sphinxstylestrong{Create} button at the bottom right of the page.

You should be redirected to the main \sphinxstylestrong{Evaluate and Deploy} page.
\begin{itemize}
\item {} 
This will start the evaluation process in which BIG-IQ compares its
working configuration to the configuration active on each BIG-IP.
This can take a few moments to complete.

\item {} 
The \sphinxstylestrong{Status} section should be dynamically updating… (What states
do you see?)

\end{itemize}

Once the status shows \sphinxstylestrong{Evaluation Complete} you can view the
evaluation results.

Before selecting to deploy, feel free to select the differences
indicated to see the proposed deployment changes. This is your check
before actually making changes on a BIG-IP.

Click the number listed under \sphinxstylestrong{Differences \textendash{} Firewall}.
\begin{itemize}
\item {} 
Scroll through the list of changes to be deployed.

\end{itemize}

Click on a few to review in more detail.

\sphinxincludegraphics[width=6.48958in,height=3.45833in]{{image125}.png}
\begin{quote}

What differences do you see from the \sphinxstylestrong{Deployed on BIG-IP} section
and on \sphinxstylestrong{BIG-IQ}?
\end{quote}

Click \sphinxstylestrong{Cancel}.

Deploy your changes by checking the box next to your evaluation
\sphinxstylestrong{deploy\_afm2}.

With the box checked, click the \sphinxstylestrong{Deploy} button.
\begin{itemize}
\item {} 
Your evaluation should move to the \sphinxstylestrong{Deployments} section.

\item {} 
After deploying, the status should change to \sphinxstylestrong{Deployment Complete}.

\item {} 
This will take a moment to complete. Once completed, log in to the
BIG-IP and verify that the changes have been deployed to the AFM
configuration.

\end{itemize}

Congratulations, you just deployed your second AFM policy via BIG-IQ!


\subsection{Locate orphaned and stale firewall rules}
\label{\detokenize{class1/module5/lab3:locate-orphaned-and-stale-firewall-rules}}\label{\detokenize{class1/module5/lab3::doc}}

\subsubsection{WORKFLOW 3: Locate orphaned and stale firewall rules}
\label{\detokenize{class1/module5/lab3:workflow-3-locate-orphaned-and-stale-firewall-rules}}
In this lab, you will be creating a report that will show you firewall
rules that do not have any network traffic matching them. You could then
consider this firewall rule stale and potentially orphaned if it is no
longer needed in your environment.

\sphinxstylestrong{Objective:}
\begin{itemize}
\item {} 
Locate firewall rules that are orphaned (unused)

\end{itemize}

\sphinxstylestrong{Lab Requirements:}
\begin{itemize}
\item {} 
Web UI access to BIG-IQ

\end{itemize}


\paragraph{Task 1 \textendash{} Review Network Firewall Security Reports}
\label{\detokenize{class1/module5/lab3:task-1-review-network-firewall-security-reports}}
On \sphinxstylestrong{BIGIQ1}: (\sphinxhref{about:blank}{*https://10.0.0.200)*}

Navigate to \sphinxstylestrong{Monitoring} from the top tabs of \sphinxstylestrong{BIG-IQ}.

In the left navigation, click on Reports\(\rightarrow\)Security\(\rightarrow\)Network
Security\(\rightarrow\)Firewall Rule Reports.

\sphinxincludegraphics[width=4.25000in,height=4.43750in]{{image126}.png}

Click \sphinxstylestrong{Create}.

For \sphinxstylestrong{Name}, type in \sphinxstylestrong{test\_report}.

On the \sphinxstylestrong{Report Type} dropdown, select \sphinxstylestrong{Stale Rule Report}.

On \sphinxstylestrong{Stale Rule Criteria}, select Rules with count less than \sphinxstylestrong{1} and
use today’s date

On Available Devices, move bigip1.agility.f5.com to the right Selected
box.

Click Save \& Close.

Click on the report name \sphinxstylestrong{test\_report}.

\sphinxincludegraphics[width=6.48958in,height=3.71875in]{{image127}.png}

Down below to the right of \sphinxstylestrong{Report Results}, click on \sphinxstylestrong{HTML Report}.
\begin{itemize}
\item {} 
There might be a browser pop up block warning in the upper right
corner of your browser.

\item {} 
Allow the pop up. You may have to click on \sphinxstylestrong{HTML Report} again.

\end{itemize}

You should now see a report of rules that do not have \sphinxstylestrong{Hit Counts}.

You can also export CSV for further processing of data by selecting
\sphinxstylestrong{CSV Report.}


\section{External Logging Devices (SevOne)}
\label{\detokenize{class1/module6/module6:external-logging-devices-sevone}}\label{\detokenize{class1/module6/module6::doc}}
BIG-IQ Central Management Version - 5.1 has introduced the ability to
integrate with SevOne’s Performance Logging Appliance (PLA) for high
speed logging and reporting.

With SevOne, you can move beyond traditional log search, into real-time troubleshooting of log data at scale. That’s because it extracts terabytes of log data daily and correlates it to performance events, thereby eliminating the need for search in your process. It also increases application performance visibility by providing single-click drill-down from related data — SNMP metrics to NetFlow records to syslog files, for example
\begin{itemize}
\item {} 
Automatically correlate real-time performance metrics with log data.

\item {} 
Improve visibility of the root cause of performance degradation.

\item {} 
Receive proactive alerts of customer and end-user behavioral trends.

\item {} 
Decrease time-to-troubleshoot with integrated metrics, flow and logs.

\item {} 
Eliminate timely error code analysis with fully configurable
value-based lookups.

\item {} 
Get targeted and intelligent anomaly detection using multiple log
metrics across many devices in real-time.

\item {} 
Leverage multi-variable alerting of log data to identify issues and
trends.

\item {} 
Gain a greater understanding of how configuration changes impact
application performance.

\end{itemize}


\subsection{Add External Logging Device}
\label{\detokenize{class1/module6/lab1::doc}}\label{\detokenize{class1/module6/lab1:add-external-logging-device}}

\subsubsection{WORKFLOW 1 : Add an External Logging Device and Configure Single Sign On}
\label{\detokenize{class1/module6/lab1:workflow-1-add-an-external-logging-device-and-configure-single-sign-on}}
In this lab, you’ll be creating a connection to an external logging
device (PLA) and configuring single sign on for monitoring

\sphinxstylestrong{Objective}
\begin{itemize}
\item {} 
Create an external logging device

\item {} 
Create a login token for SSO

\end{itemize}

\sphinxstylestrong{Lab Requirements}
\begin{itemize}
\item {} 
Web UI access to BIG-IQ

\end{itemize}


\paragraph{Task 1 \textendash{} Create an external logging device}
\label{\detokenize{class1/module6/lab1:task-1-create-an-external-logging-device}}
Log into BIG-IQ at \sphinxhref{https://10.0.0.200}{*https://10.0.0.200*}

Navigate to \sphinxstylestrong{System} from the top tabs of \sphinxstylestrong{BIG-IQ}.

In the left navigation, click on \sphinxstylestrong{BIG-IQ Data Collection} \(\rightarrow\)
\sphinxstylestrong{3:sup:{}`rd{}` Party Data Collection Devices}

\sphinxincludegraphics[width=5.80000in,height=5.40000in]{{image128}.png}

Click Add to add a new 3$^{\text{rd}}$ Party Data Collection Device

\sphinxincludegraphics[width=6.50000in,height=6.73958in]{{image129}.png}

Complete the page with the following table:


\begin{savenotes}\sphinxattablestart
\centering
\begin{tabulary}{\linewidth}[t]{|T|T|T|}
\hline

\sphinxstylestrong{Name}
&
\sphinxstylestrong{Lab\_PLA}
&\\
\hline
{\color{red}\bfseries{}**}Description **
&
\sphinxstylestrong{LAB PLA}
&\\
\hline
\sphinxstylestrong{Device Type}
&
\sphinxstylestrong{SevOne PLA}
&\\
\hline
\sphinxstylestrong{IP Address}
&
\sphinxstylestrong{10.128.10.202}
&
Check Use as Query Server
\\
\hline
\sphinxstylestrong{User Name}
&
\sphinxstylestrong{root}
&\\
\hline
\sphinxstylestrong{Password}
&
\sphinxstylestrong{dRum\&5853}
&\\
\hline
\sphinxstylestrong{Push Schedule}
&&\\
\hline
\sphinxstylestrong{Status}
&
\sphinxstylestrong{Enabled Checked}
&\\
\hline
\sphinxstylestrong{Push Frequency}
&
\sphinxstylestrong{Daily}
&\\
\hline
\sphinxstylestrong{Start/End Date}
&
\sphinxstylestrong{Set to beginning of tomorrow’s date (00:00)}
&\\
\hline
\end{tabulary}
\par
\sphinxattableend\end{savenotes}

When completed click the “Test” button on the “Test Connection”

Test will come back successful if settings are all correct as shown
below

\sphinxincludegraphics[width=5.69444in,height=4.95139in]{{image130}.png}

Click Add to complete the addition of an external logging device.


\paragraph{Task 2 \textendash{} Configure Single Sign On}
\label{\detokenize{class1/module6/lab1:task-2-configure-single-sign-on}}
Navigate to \sphinxstylestrong{Monitoring} from the top tabs of \sphinxstylestrong{BIG-IQ}.

On the left navigation, click on \sphinxstylestrong{Events}\(\rightarrow\)\sphinxstylestrong{Network
Security}\(\rightarrow\)\sphinxstylestrong{3:sup:{}`rd{}` Party Data Collection Devices}.

\sphinxincludegraphics[width=6.50000in,height=3.33333in]{{image131}.png}

Click on “Request Auth Token”. This will bring up the SevOne PLA
Authentication Token screen

\sphinxincludegraphics[width=5.13889in,height=2.33333in]{{image132}.png}

Fill in \sphinxstylestrong{*admin*} for username and \sphinxstylestrong{*SevOne*} for the password
and click “Request Token”

If the values are correct, a token will be returned

\sphinxincludegraphics[width=5.13889in,height=2.36806in]{{image133}.png}

Click “Save” to save the configuration changes.

You can now click on the “Launch” link to log into the PLA without
having to supply a username and password.

Additional Resources:

\sphinxurl{https://support.f5.com/kb/en-us/products/big-iq-centralized-mgmt/manuals/product/bigiq-central-mgmt-security-5-2-0/10.html\#guid-8dbb4024-a82e-4173-83b0-72e0365207e4}


\subsection{Login to Your PLA}
\label{\detokenize{class1/module6/lab1:login-to-your-pla}}
To login to your PLA open up a browser on your laptop and use the
Management IP address (\sphinxstylestrong{10.128.10.202}) or use the Single Sign on in
BIG-IQ.

i.e. \sphinxurl{https://10.128.10.202}

The login information for your PLA is \sphinxstylestrong{*admin/SevOne*}


\subsection{Familiarize yourself with PLA Settings page}
\label{\detokenize{class1/module6/lab2:familiarize-yourself-with-pla-settings-page}}\label{\detokenize{class1/module6/lab2::doc}}

\subsubsection{WORKFLOW 1: Familiarize yourself with PLA Settings page}
\label{\detokenize{class1/module6/lab2:workflow-1-familiarize-yourself-with-pla-settings-page}}
In this lab, you’ll be reviewing the Settings page within PLA.

The Settings page can be accessed by logging into your PLA and clicking
on the settings at the top of the page

\sphinxincludegraphics[width=3.08333in,height=0.62500in]{{image134}.png}

Each of the sections is briefly detailed below:

Personal Preferences \textendash{} Changes personal information, date display format
and your background.

User Management \textendash{} Create, enable, and disable users as well as their
access levels.

Data Inputs \textendash{} Create, monitor and stop/start data input listeners.

Cluster Management \textendash{} View cluster statistics and storage information.

Container Management \textendash{} View and create new containers and their
respective data.

Active Queries \textendash{} View and control actively running query processes.

Processing Jobs \textendash{} View and control processing tasks in the job queue

Application Keys \textendash{} Manage application key files.

Tuples Management \textendash{} Manage Tuple Tag lists

Alert Configuration \textendash{} Define alerting rules applied to incoming data

Data Retention \textendash{} Configure data retention options and view average
usage.

Licensing \textendash{} Manage the license associated with the PLA

Software Update \textendash{} Check for software updates.


\subsection{Creating and Viewing Alerts}
\label{\detokenize{class1/module6/lab3:creating-and-viewing-alerts}}\label{\detokenize{class1/module6/lab3::doc}}

\subsubsection{WORKFLOW 2: Creating and Viewing Alerts}
\label{\detokenize{class1/module6/lab3:workflow-2-creating-and-viewing-alerts}}
In this lab, you’ll be creating a new alert and reviewing the alert

Navigate to \sphinxstylestrong{*Settings*} \textgreater{} \sphinxstylestrong{*Alert Configuration*}

Click “\sphinxstylestrong{*Add New Rule*}”

Select Content text matching or context tag analytics based on the rule
you are trying to create. For example, to create a new rule that will
alert when a new IP is ingested into the system (E.G. for a new DDoS
alerting mechanism) you would enter the following:

Ruleset \textendash{} Context Tag Analytics

Data Examination Method \textendash{} First Value Occurrence (FVO is unique to the
PLA)

Tags \textendash{} SrcIP

Alert Thresholds \textendash{} Upper 1 Lower 0

Alert Options \textendash{} Enable (email or trap) or if left to \textendash{}No Script—Alerts
will appear on the Alerts Page

\sphinxincludegraphics[width=2.02736in,height=4.50618in]{{image135}.png}

When finished click +\sphinxstylestrong{Add} and you will see a summary of the current
alerts

\sphinxincludegraphics[width=6.50000in,height=1.38194in]{{image136}.png}

If no external alerting mechanism is configured you can view current
alerts on the \sphinxstylestrong{*Alerts*} tab as shown:

\sphinxincludegraphics[width=6.50000in,height=1.04167in]{{image137}.png}

Now whenever a new Source or Destination IP is processed by the PLA an
alert will be created.


\subsection{Viewing event logs and creating a tuple}
\label{\detokenize{class1/module6/lab4:viewing-event-logs-and-creating-a-tuple}}\label{\detokenize{class1/module6/lab4::doc}}

\subsubsection{WORKFLOW 3: Viewing event logs and creating a tuple}
\label{\detokenize{class1/module6/lab4:workflow-3-viewing-event-logs-and-creating-a-tuple}}
In this lab, you’ll become familiar with the Investigate section, this
is where the raw logs can be viewed along with tuples created.

Navigate to the \sphinxstylestrong{*Investigate*} tab at the top of the page.

Verify you’re looking at the most recent data by selecting the past 2
hours from the hamburger next to the date near the top of the screen as
shown below

\sphinxincludegraphics[width=6.50000in,height=1.14583in]{{image138}.png}

Select a field tag to view the data within the tag:

\sphinxincludegraphics[width=2.07153in,height=1.92847in]{{image139}.png}

The far right will display the values within the selected tag:

\sphinxincludegraphics[width=2.24306in,height=1.29167in]{{image140}.png}

This is useful if you are looking for data from a specific device for
troubleshooting.

Click on the value and data will load from that selected field in the
viewing pane:

Different representations of the data can be selected by toggling
through the graph, textual and grid-style as shown below

\sphinxincludegraphics[width=6.50000in,height=1.55556in]{{image141}.png}

\sphinxincludegraphics[width=6.49306in,height=1.69444in]{{image142}.png}

\sphinxincludegraphics[width=6.49306in,height=2.78472in]{{image143}.png}

To view the logs from Lab one

Click on “Investigate”

From the Central Menu select the F5-Network-Firewall container as shown
in the following image:

\sphinxincludegraphics[width=5.56875in,height=2.25278in]{{image144}.png}

From left Menu under \sphinxstylestrong{*Event Tag*} click on \sphinxstylestrong{*application.*}

From the Right Menu click review the logs in the central part of the
screen.

To select different containers or compound key searches select the
desired keys from the drop down to represent the data you are
investigating:

\sphinxincludegraphics[width=4.28472in,height=4.00000in]{{image145}.png}

As more data is selected more tags become available for further
analysis.

Tuples allow for quick views of multiple tags \textendash{} for example if you
wanted to always view just the srcIP, destIP, destPort, action and
hostname you could build a quick tuple for this data representation.

To create a new Tuple, navigate to Setting \textgreater{} Tuples Management

Click Add to create a new tuple and select the desired tags along with
the time interval.

\sphinxincludegraphics[width=4.72222in,height=4.25000in]{{image146}.png}

When finished click save.

Tuples can be viewed from the \sphinxstylestrong{*Investigate*} page under tuple tags
(note tuples take 5 minutes to refresh their defined data):

\sphinxincludegraphics[width=2.03472in,height=0.77083in]{{image147}.png}


\subsection{Creating Reports}
\label{\detokenize{class1/module6/lab5::doc}}\label{\detokenize{class1/module6/lab5:creating-reports}}

\subsubsection{WORKFLOW 4: Creating Reports}
\label{\detokenize{class1/module6/lab5:workflow-4-creating-reports}}
Navigate to Investigate Page \& Select Past Two Hours

\sphinxincludegraphics[width=4.77778in,height=0.33738in]{{image148}.png}

\sphinxincludegraphics[width=2.36875in,height=2.11458in]{{image149}.png}\sphinxincludegraphics[width=4.18627in,height=0.90821in]{{image150}.png}

Right click ‘action’ column header to filter column, click on host \&
drag filter to select = (type in drop)

\sphinxincludegraphics[width=6.50000in,height=4.30000in]{{image151}.png}

\sphinxincludegraphics[width=6.50000in,height=3.00000in]{{image152}.png}

Click the “Puzzle” piece icon to open Report Attachment Editor

\sphinxincludegraphics[width=6.50000in,height=2.01042in]{{image153}.png}

Enter Parameters:
\begin{itemize}
\item {} 
Title

\item {} 
Moving time frame

\item {} 
Report Page

\end{itemize}

\sphinxincludegraphics[width=5.91990in,height=2.63426in]{{image154}.png}

Navigate to Reports Page

Click on “Gear” icon and select Show Bar Chart

\sphinxincludegraphics[width=6.50000in,height=3.30000in]{{image155}.png}

The appropriate chart selected will be displayed

\sphinxincludegraphics[width=5.59375in,height=5.05208in]{{image156}.png}



\renewcommand{\indexname}{Index}

\backcoverpage

\end{document}